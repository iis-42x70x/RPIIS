\documentclass[twocolumn, 10pt]{article}
\usepackage{graphicx} % вставка картинок
\usepackage{multicol} %колонки
\usepackage[colorlinks=true, urlcolor=red]{hyperref} % гиперссылки
\usepackage[english, russian]{babel} % для переключения языков
\usepackage{fancyhdr} % используется для настройки колонтикулов

\usepackage[left=19mm, top=22mm, right=19mm, bottom=20mm,]{geometry}
\usepackage{enumitem} % списки
\setlength{\columnsep}{0.5cm} % расстоояние между колонками
\graphicspath{{images/}} % для указания местоположения изображений
\setcounter{page}{309} % устанавливает нужный номер страницы
\setcounter{figure}{9} %устанавливает номер фигуры
\usepackage{setspace} % выравнивания текста по ширине в документе.
\linespread{1} %устанавливает межстрочный интервал
\fancyfoot[C]{\bfseries \thepage} % устанавливает жирные номера страниц
\renewcommand{\thesection}{\Roman{section}}
\setcounter{section}{6}
\newcommand{\RomanNumeralCaps}[1]
    {\MakeUppercase{\romannumeral #1}}





\begin{document}

\selectlanguage{english} % устанавливает английский язык
\noindent stable learning and good results when using sequential %убирает красную строку
learning, where one client is trained first and then the
other. However, by using federated learning methods, we
were able to achieve the same metric values as in the
case of balanced data. This is primarily because, during
aggregation, the global model aims to acquire knowledge
from all clients and average them.

Each of the client models learned to predict the classes
present in its own dataset very well. However, when it
comes to predicting classes that were rarely encountered
before, the client models struggle on the test data.
Nonetheless, the central model aggregated information
from all the data, which can potentially improve its ability
to predict such classes.

Thus, federated learning methods enable us to mitigate
the impact of data heterogeneity when training models.

In the figure 10 is shown the graph of accuracy for
heterogeneous data

\begin{figure}[h] % добавляет картинку
    \centering
    \includegraphics[width=8.5cm]{photo_2024-09-10_22-23-58}
    \caption{ Graph of accuracy for heterogeneous data.} 
\end{figure}


\centerline{ VI. Semantic technologies application}

In addition to the above applications, you can use
semantic technologies in federated learning. For example
as follows.
\begin{itemize} % создание ненумерованного списка
    \item Distributed knowledge representation: Semantic
    technologies enable the representation of knowledge
    in a structured manner using formal languages. This
    can be useful in distributed model training, where
    each device can have its local knowledge representation and then combine these representations on a
    central server
    \item Semantic data analysis: Semantic technologies can
    assist in the analysis and understanding of data
    collected from distributed devices. For example,
    they can be used to extract meaning and relationships between data, which can be beneficial for
    aggregation and merging of models on a central
    server.
\end{itemize}
\newpage % создание новой страницы
\begin{itemize}
    \item Unification of semantic understanding: Semantic
    technologies can help unify the understanding of
    data across different devices or servers. They can
    be used to create a shared model of knowledge
    or a semantic network, which can be utilized for
    harmonizing and collaboratively training models on
    different devices.
\end{itemize}
\begin{center}
\RomanNumeralCaps{7}.  Conclusion
\end{center}

In this article, the main problems arising in the task of
biomedical image analysis were described, and a method
to avoid them was proposed. Furthermore, experiments
were conducted to demonstrate its practical applicability.
Has been shown that federated learning helps preserve
data confidentiality and also provides a significant improvement in quality when different clients have data
from different classes. Various approaches to solving the
problem of heterogeneous learning have been considered.
The obtained conclusions and recommendations can be
valuable for researchers in the field of biomedical informatics and medicine who aim to utilize advanced
machine learning methods for image analysis in privacypreserving conditions.
\begin{center}
    Acknowledgment
\end{center}

This research was supported by the United Institute
of Informatics Problems of the National Academy of
Sciences of Belarus (UIIP NASB).
\begin{center}

\begin{thebibliography}{10} %создание списка литературы
\fontsize{8}{5}\selectfont %усианавливает размер шрифта
\setlength{\parindent}{0.0cm} % выравнивает абзацы
\setlength{\parskip}{0.0cm} % убирает расстояние между абзацами
    \bibitem{}
    H. Brendan McMahan, E. Moore, D. Ramage, S.Hampson, B.
    Agüera y Arcas, “Communication-efficient learning of deep
    networks from decentralized data,” Artificial Intelligence and
    Statistics, 1273–1282 (2017).
    \vspace{0mm}
    \bibitem{}
    P. Qi, D. Chiaro, A. Guzzo, M. Ianni, G. Fortino, F. Piccialli
    “Model aggregation techniques in federated learning: A comprehensive survey”. Future Generation Computer System, 150, 272-
    293 (2023).
    \bibitem{}
    Federated Learning for Mobile Keyboard Prediction, McMahan,  Brendan, et al., 2017.
    \bibitem{}
    EU. Regulation (eu) 2016/679 of the european parliament and
    of the council on the protection of natural persons with regard
    to the processing of personal data and on the free movement of
    such data, and repealing directive 95/46/ec (general data protection regulation). \href{https://eur-lex.europa.eu/legal-content/EN/TXT.} {https://eur-lex.europa.eu/legal-content/EN/TXT.},
    2018. % создание гиперссылки
    \bibitem{}
    Brendan McMahan, Eider Moore, Daniel Ramage, Seth Hampson,
    and Blaise Agüera y Arcas. Communicationefficient learning
    of deep networks from decentralized data. In Aarti Singh and
    Xiaojin (Jerry) Zhu, editors, Proceedings of the 20th International
    Conference on Artificial Intelligence and Statistics, AISTATS
    2017, 20-22 April 2017, Fort Lauderdale, FL, USA, volume 54
    of Proceedings of Machine Learning Research, pp. 1273–1282.
    PMLR, 2017.
    \bibitem{}
    Kairouz Peter, McMahan H Brendan, Avent Brendan, Bellet
    Aurélien, Bennis Mehdi, Bhagoji Arjun Nitin, Bonawitz Keith,
    Charles Zachary, Cormode Graham, Cummings Rachel, et al.
    Advances and open problems in federated learning. arXiv preprint
    arXiv:1912.04977, 2019.
    \bibitem{}
    Hard Andrew, Rao Kanishka, Mathews Rajiv, Ramaswamy Swaroop, Beaufays Françoise, Augenstein Sean, Eichner Hubert, Kiddon Chloé, and Ramage Daniel. Federated learning for mobile keyboard prediction. arXiv preprint arXiv:1811.03604, 2018.
    \clearpage
    \bibitem{}
    Jin Yilun, Wei Xiguang, Liu Yang, and Yang Qiang. A survey towards federated semi-supervised learning. arXiv preprint
    arXiv:2002.11545, 2020.
    \bibitem{}
    Tian Li, Anit Kumar Sahu, Ameet Talwalkar, and Virginia Smith.
    Federated learning: Challenges, methods, and future directions.
    IEEE Signal Process. Mag., 37(3):50–60, 2020.
    \bibitem{}
    Zengpeng Li, Vishal Sharma, and Saraju P. Mohanty. Preserving
    data privacy via federated learning: Challenges and solutions.
    IEEE Consumer Electronics Magazine, 9(3):8–16, 2020.
\end{thebibliography}
\end{center}

\selectlanguage{russian} %устанавливает русский язык
\begin{center}
\subsubsection*{\textbf{ПРОБЛЕМЫ \\ КОНФИДЕНЦИАЛЬНОСТИ И \\ НЕОДНОРОДНОСТИ ПРИЛОЖЕНИЙ \\ ФЕДЕРАТИВНОГО ОБУЧЕНИЯ ПРИ \\ АНАЛИЗЕ МЕДИЦИНСКИХ \\ ИЗОБРАЖЕНИЙ}}
\footnotesize{Гимбицкий А. В., Зеленковский В. П., \\ Жидович М. С., Ковалёв В. А.}
\end{center}

В последнее время машинное обучение стало одним
из самых многообещающих направлений в работе с
медицинскими данными. Модели глубоких нейронных
сетей являются наиболее эффективными и точными,
но требуют больших объемов информации для обучения. Это общая проблема в случае медицинских
данных, особенно изображений, так как их создание
включает значительные затраты. Одним из решений
для повышения качества моделей глубокого обучения
без увеличения обучающего набора данных является агрегация моделей. Однако возникает проблема
сохранения конфиденциальности медицинских изображений. Например, если одна модель обучается на
изображении, содержащем информацию о конкретном
пациенте, другие модели, участвующие в агрегации,
также могут получить доступ к этой информации. В
результате информация о конкретном пациенте может
быть раскрыта.

В попытке решить описанную выше проблему, данная работа направлена на исследование и разработку
методов агрегации моделей машинного обучения с
сохранением конфиденциальности медицинских изображений, в особенности методов федеративного обучения.

\selectlanguage{english}
\hfill{Received 13.03.2024} % выравнивает текст по правому краю
\begin{onecolumn}
\begin{center}
    \fontsize{22}{30}\selectfont
    {\textbf{Evaluation Metrics and Multi-level GAN \\ Approach for Medical Images}}
\end{center}
\begin{center}  
    Galina Kovbasa
    
    \textit{Belarusian State University of Informatics and Radioelectronics}

    Minsk, Belarus

    g.kovbasa@gmail.com

\end{center}
\end{onecolumn}

\begin{multicols}{2}

\small{\textbf{\textit{Abstract}—This article examined methods for using GANs
in medicine, their prospects, as well as problems with
training generative adversarial networks associated with
the increasing use of generated images for training other
networks. The analysis of single-layer and multi-layer GANs
concluded that although multi-layer GANs perform better
statistically, they do not exactly match the distribution
of the original dataset and, without medical supervision,
such synthetic data should not be used when training new
networks. Problems associated with the phenomenon of
recursive learning, biased assessments of image realism,
and non-optimized structures are considered. Approach is
described in context of integrating generative adversarial
network models into the OSTIS Technology based hybrid
computer systems.}

\textbf{\textit{Keywords}—Multi-level GANs, recursive learning, synthetic data}}

\normalsize 
\centerline{I. Introduction}

Generative adversarial networks (GANs) are a remarkably popular technique for generating realistic synthetic
data. Modern GANs can have different layers, backgrounds , complexity and be trained by semi-supervised
and unsupervised learning. They gain their popularity because of the ability to implicitly modeling highdimensional data distributions. [1] GANs are of particular interest in the processing, classification and evaluation
of medical images, in the future making it possible to
speed up and improve the analysis of the results of
magnetic resonance imaging, computed tomography, Xrays and others. This can be solved by integrating a GAN
neural networks into the OSTIS Technology. [2]

Integration with third-party technologies based on
neural networks allows the development of universal
hybrid systems. GANs are capable of not only processing
images, but also creating new synthetic data on demand, which makes them valuable for creating datasets,
anonymous educational materials, etc. GANs are actively
changing during development and all these changes can
be formalized in the OSTIS system using SC code. [3]

Thus, network artifacts of the processes of creation,
training and further tuning, such as numbers and types
of layers, weights, activation functions, can be stored in a
general form and saved, supporting further replenishment
and expansion of the general knowledge base. [4] The
OSTIS intelligent system is also capable of saving several
\columnbreak


{\centering
\includegraphics[width=9cm]{photo_2024-09-10_22-24-16.jpg}

\caption{\small{Figure 1. Model collapsing. Over generations, the generated data begins to look unimodal. \hfill}}}
\setlength{\parskip}{1.0cm}

\noindent versions of the same model for later use, even restoration
from a previous point. [5]
\setlength{\parskip}{0.0cm}

But one of the worst challenges facing GANs in
medicine is that medical images are susceptible to various noise and artifacts common to different modalities
and, what most important is, have a very little variety of
datasets to be used. It should also be noted that much
medical information is 3D structures, which can make it
difficult to train a GAN on only 2D images.

\subsubsection*{\mdseries{\textit{A. The recursive learning problem}}}


No generative adversarial network can reliably recreate
the distribution of the original sample data. It may make
mistakes or recreate the same data, that is, clone it. We
must train the network so that it does not go beyond the
distribution, but also does not repeat the picture. This
solution allows us to avoid modern problems with GAN.

Generative adversarial networks are able to generate
high-quality images on demand using the same distribution that is given, the score is also high. However, this
does not mean that GAN can be used as a universal
method or classifier, since it is impossible to assume
that the distribution will be reliable or true. Likewise,
the images produced by networks cannot be considered
representative for training other networks.

However, more and more people are using GANs for
reconstruction and generation, but this only leads to the
fact that it can be used for a training set, since the
generated materials are in the public domain. As a result,
the networks degenerate, and the results of reconstruction
and generation deteriorate. And as a result, the so-called
mode collapse occurs (Fig. 1). [6]

This phenomenon (using generated images as training materials) is most dangerous for image generation,




\end{multicols}
\end{document}


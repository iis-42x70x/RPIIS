\documentclass{article} 
\usepackage{multicol} 
\usepackage{float} 
\usepackage[left=2cm,right=2cm,top=1cm,botton=3cm]{geometry} 
\usepackage{parskip} 
\setcounter{page}{141} 
\begin{document} 
\setlength{\parskip}{0.023cm}
\begin{multicols}{2}

cific knowledge satisfying clearly defined requirements
is placed, significantly simplifies navigation through the
knowledge base and clearly localises the evolution of
knowledge located on each "semantic shelf".From a
meaningful point of view, the subject domain is a set
of factual statements describing all elements of a given
set of objects of research with the help of a given set of
relations and parameters (characteristics). 

\quad
A section - is a sign of a set of all possible sections
included in different knowledge bases. Each section repre-
sents a conditionally didactically distinguished fragment
of the knowledge base, possessing logical integrity and
completeness. In the limit, the whole knowledge base
of a particular ostis-system is also one large non-atomic
section.

\quad
For each partition it is necessary to explicitly specify
the belonging to a set of atomic or non-atomic partitions.
Atomic partition — a sign of the set of all possible atomic
partitions included in various documentation, i.e., partitions not decomposable into more private partitions. Non-atomic section — the sign of the set of all possible non-
atomic sections that make up the various documentations,
that is, sections that are decomposed into more private
sections.

\quad
In the context of the \textit{OSTIS Glossary}, it may be appro-
priate to identify sections that will describe some portion
of that information described in the relevant sections
of the \textit{OSTIS Standard}. But nobody forbids to consider
the \textit{OSTIS Glossary} as some atomic section in which
all objects are listed according to the external natural
language alphabet together with their specifications. As
for subject domains, it is more expedient to use them
not as a means for knowledge stratification, but as a
means for searching already existing information to form
sections with this information.


\quad{So, let’s list the \textbf{\textit{basic rules of structuring and
specification of the OSTIS Glossary objects}:}}

\begin{itemize}

\item{\textmd{ All information about the \textit{OSTIS Glossary} objects
should be represented in the form of semantic
neighbourhoods of these objects, in which will be
listed:}}
\begin{itemize}
\item[-] identification variants (variants of terms) of a
given object in various external (natural) lan-
guages;
\item[-] membership of the object in some subject domain
with indication of the role performed within this
subject domain, including membership of this
object in the corresponding section of the knowl-
edge base or the \textit{OSTIS Standard};
\item[-] theoretical-multiple relations of the given object
with other objects, including:
\begin{itemize}
\item[*] partition and decomposition of a given object into other objects;
\item[*] including a given into other objects;
\end{itemize}
\end{itemize}
\end{itemize}

\begin{itemize}
\begin{itemize}
\item[*] and other possible unions, intersections  
that form a given object;
\end{itemize}
\item[-] definition of the given object; 
\item[-] explanation of the given object;
\item[-] description of a typical example of using the
specified object;
\item[-] instances of the described object, if the given
object is a concept;
\item[-] authors of the given object;
\item[-] authors of the specification, i.e. the authors who
described the given object in the knowledge base;
\item[-] analogues of this object, including:
\end{itemize}

\begin{itemize}
\begin{itemize}
\item[*]close analogues of the given object;
\item[*]differences and similarities with other 
objects, including listed analogues;
\end{itemize}
\item[-] bibliographic sources of the object;
\item[-] possible quotations, aphorisms, metaphors and
epigraphs related to the given object; This variant
of specification will be called dictionary specifi-
cation.

\hspace{-6mm}• In this case, all objects and their specifications

\hspace{-3mm}within the \textit{OSTIS Glossary} can be ordered:

\item[-] as an enumeration of these objects in the lex-
igraphic order of their terms, forming one sin-
gle atomic section, respectively being of the \textit{OSTIS
Glossary}; 
\item[-] and the enumeration of sections corresponding
to their sections of the \textit{OSTIS Standard}, uniting unique
objects by the common features, considered within
a particular section, and representing sequences
of these objects in the lexigraphic order of their
terms. 

\hspace{-5mm} For the second case, it is important that the 

\hspace{-5mm} sections formed sections of the \textit{OSTIS} Glossary.

\hspace{-5mm} do not overlap with each other, i.e., do not 

\hspace{-5mm} duplicate descriptions of the same entity. 

\hspace{-5mm} From this point of view, the first option is easier 

\hspace{-5mm} to implement, because it does not require taking 

\hspace{-5mm} into account the possibility of occurrence of the 

\hspace{-5mm} found object in the already formed sections of 

\hspace{-5mm} the \textit{OSTIS} Glossary.

\hspace{-8mm} • Structuring of objects of the \textit{OSTIS} Glossary 

\hspace{-4mm}be reduced to the formation of text from the

\hspace{-4mm}already structured text of the \textit{OSTIS} Standard.
\end{itemize}

\textit{D. Rules for identification of the \textit{OSTIS} Glossary} 

\quad 
In the previous section the rules of structuring and
specifying (standardisation) of the \textit{OSTIS Glossary} ob-
jects were fixed. However, it is necessary to standardise
not only those texts that are written directly in \textit{SC-code}, but also those texts that are written in external, e.g.
natural languages. One of features it is also possible to record some natural
language files, the comprehension
of the text by any human being is improved. Such a
possibility is realised with the help of ostis-system files,

\end{multicols}
\begin{multicols}{2}

with the help of which external information constructions
which are not text in \textit{SC-code} [43] are denoted. With their
help it is possible to specify information in an external
natural language for all objects in the knowledge base.
A special case of ostis-system files are files denoting
identifiers of objects in the knowledge base.

\quad 
Identifier is a structured sign representing an entity
denoted by a string of symbols. An identifier is an
information construct (most often a string of symbols)
providing unambiguous identification of the correspond-
ing object described in knowledge bases of ostis-systems,
texts of ostis-systems.

\quad
In formal texts, identifiers must be unique to uniquely
match an object. Each pair of identical identifiers must
denote the same object.

\quad
All objects in a knowledge base have the following

\textbf{\textit{common identification rules}:}
\begin{itemize}
\item[•]The membership of the identified object in some
object classes is explicitly specified in the external
identifier of this object (in the sc-identifier) by
means of appropriate conditional attributes:
\begin{itemize}
\item if the last character of the sc-identifier is an “aster-
isk” character, then the identified object belongs
to the Class of non-role relation designations;

\item if the last character of the sc-identifier is an
apostrophe, then the identified object belongs to
the Class of role relationship designations, each
of which is a subset of Membership relation

\item if the last character of the sc-identifier is “”,
then the identified object belongs to the Parameter
designation class.
\end{itemize}
\item[•]For each object, you can construct an sc-identifier,
which is a proper name that always starts with a
capital letter.
\item[•]If an object is a designation of some class of objects,
then this object can be matched not only with a
proper name, but also with a common name, which
starts with a small (lowercase) letter. The specifi-
cation of each class (each concept) includes a list
of equivalent (synonymous) sc-identifiers, among
which there are both proper and nominative names.
\item[•]Identification of partitions in knowledge bases is
performed at the expense of identification of par-
tition objects. Instances of partition classes within
the Russian language are named according to the
following rules:
\begin{itemize}
\item at the beginning of the identifier the word Section
is written and a dot is put;
\item followed by the name of the section with a capital
letter, reflecting its content.
\end{itemize}
\end{itemize}

Obviously, the same rules apply to the \textit{OSTIS Glossary}
objects. Therefore, there is no need to describe any addi-

tional rules for identifying the \textit{OSTIS Glossary} objects.

\quad
Besides identifiers of ostis-systems knowledge base ob-
jects, it is possible to standardise the form of presentation
of information in the specification of these objects: both
the style of writing the text itself and the information
with the help of which it is possible to explain the basic
information in the specification of these objects.

\textit{E. Key elements of didactic information in the OSTIS
Glossary object specifications}

\quad
The most important criterion of quality of created
ostis-systems of any purpose is to create conditions so
that insufficiently qualified users of each ostis-system
(both end-users and those responsible for its effective
operation and modernisation) could acquire the required
qualification quickly enough with the help of the same
ostis-system. This means that each ostis-system, irrespec-
tive of its direct purpose (automation of specific types of
human activities in a particular field) should also be a
training system, i.e. it should be able to train its users in
the direction of improving their qualification. A qualified
user of any category must understand the capabilities of
the ostis-system with which he interacts, must understand
what the system knows and can do, as well as how
its activities can be managed. Lack of understanding
between ostis-systems and their users — is a violation
of the interoperability requirement [44] imposed on both
ostis-systems and their users.

\quad
Therefore, the key stage in the development of any
knowledge base, and, in general, any information re-
source is the stage of development and implementation
of didactic information. Didactic information should be
understood as a specification of the subject domain,
which provides additional information designed to enable
users and developers (knowledge engineers), who use or
improve the specified subject domain and its ontology, to
learn their features faster. Didactic information enables
[45]:

\begin{itemize}
    
\item[•]to quickly and adequately assimilate the denotational
semantics of knowledge stored in the system;
\item[•]to provide a deeper understanding and assimilation
of the meaning of various kinds of entities (includ-
ing various knowledge);
\item[•]to establish mutual understanding between systems
and their users;
\item[•]to accelerate the process of formation of the re-
quired qualification of users in various fields.
\end{itemize}
The "didactic" effect of didactic information is pro-
vided by:
\begin{itemize}
\item[•]by sufficient detail of the studied entity (complete-
ness of the semantic neighbourhood describing the
relations of this entity with other entities)
\item[-]decomposition of the entities under consideration;
\item[-]by specifying analogues (similar entities in differ-
ent senses);

\end{itemize}
\end{multicols}

\begin{multicols}{2}

\begin{itemize}
\begin{itemize}

\item[-]indicating metaphors (epigraphs);
\item[-]indicating antipodes (entities that differ in differ-
ent senses);
\end{itemize}
\item[•]exercises — solutions to various problems using the
entities studied;
\item[•]references to knowledge stored within the same
knowledge base;
\item[•]references to bibliographic sources.

\end{itemize}

\quad
Following one of the goals of the \textit{OSTIS Glossary},
namely to provide a quality and understandable text for
the end user, it is necessary to describe and record the
\textbf{\textit{main elements of didactic information for the objects
of the OSTIS Glossary}}. Such elements \underline{should} include:

\begin{itemize}

\item[•]information by means of which the basic informa-
tion in the \textit{OSTIS Glossary} object specification is
defined or explained, i.e.:
\begin{itemize}
\item[-]various types of definitions, explanations, and
annotations for that object;
\item[-]examples of how to use this object;
\item[-]instances of this object, if it is a concept;
\end{itemize}
\item[•]information describing distinctive and similar char-
acteristics between the \textit{OSTIS Glossary} objects, in-
cluding:
\begin{itemize}
\item[-]analogies, correspondences, corollaries between
objects;
\item[-]differences and similarities between objects;

\end{itemize}
\end{itemize}

\begin{itemize}

\item[•]information that supports the significance, scientific
novelty, and practical applicability of the \textit{OSTIS Glossary} object, including:
\begin{itemize}
\item[-]of the authors of these objects, as well as the
authors who specified this object;
\item[-]bibliographic sources of these objects;
\item[-]quotations, aphorisms, metaphors and epigraphs
related to the given object;
\item[-]other.

\end{itemize}
\end{itemize}

\quad
It should be noted that the quality of the information
described in the knowledge base directly depends on the
quality of the presentation of this information and the
quality of the information with the help of which the
basic information in the knowledge base is explained.
The more qualitatively the information in the knowledge
base is described, structured and stratified, the lower the
requirements to the readers of this information and the
higher the level of understanding of the information in
the knowledge base by these readers [46].

\quad
Didactic information in the knowledge base determines
the level of quality of the information described in the
knowledge base. The key elements of didactic informa-
tion are those elements that contribute to easier and
deeper mastering of the basic information in the knowl-
edge base. The more qualitatively the information in the knowledge. The authors believe that it is the information
that reveals similarities between objects in the knowledge
base that is the key to improving the quality of the entire
knowledge base.

\textit{F. Conclusion}

\quad
To summarize, it is important to note the following
points:

\begin{itemize}

\item[•]The \textit{development of the OSTIS Glossary} is reduced
to the development of the \textit{OSTIS Standard} and a set
of tools that allow to form and improve the \textit{OSTIS Standard} on the basis of the \textit{OSTIS Standard};
\item[•]When developing the \textit{OSTIS Glossary} as part of the
\textit{OSTIS Standard}, it is important to fix and improve
the principles and rules:
\begin{itemize}
\item[-]specification and formalisation of objects;
\item[-]identification of objects and their specifications;
\item[-]of structuring and stratification of object specifi-
cations;
\item[-]development and consistency of objects and their
specifications;
\end{itemize}
\item[•]The \textit{quality of the OSTIS Glossary} is determined by:
\begin{itemize}
\item[-]quality of the \textit{OSTIS Standard}, which is defined
by:
\begin{itemize}
\item[*]the quality of the information it contains;
\item[*]the quality of the means to improve this
information, which is determined by:
\begin{itemize}
\item[·]the quality of the methods, principles and
rules for developing this information;
\item[·]the quality of didactic information explain-
ing this information;
\end{itemize}
\item[*]the competence and level of training of the
developers of the \textit{OSTIS Standard}.

\end{itemize}
\end{itemize}
\end{itemize}

\quad
In the final section the current state of the \textit{OSTIS
Glossary} within the \textit{OSTIS Metasystem} will be consid-
ered, the principles of automatic formation of the \textit{OSTIS
Glossary}, information retrieval in it will be fixed, and
also the current Author team of the \textit{OSTIS Glossary} and
requirements to its developers will be considered.

\begin{center}

\subsubsection*{\textmd{\normalsize{IV. Implementation of the \textit{OSTIS Glossary} within the
\textit{OSTIS Metasystem}}}}
 
 \end{center}

\textit{A. Current specification and structure of the OSTIS
Glossary}

\quad
It is important to note that the \textit{OSTIS Glossary} is not
just a dynamically generated text from the text of the
\textit{OSTIS Standard}, which is a simplified representation of
the \textit{OSTIS Standard}. The \textit{OSTIS Glossary} acts as a means
to provide a consistent and interoperable activity for
the development of new generation intelligent computer
systems, and thus has documentation for its use and
development. Therefore, first of all, the \textit{OSTIS Glossary}
is documentation on how to property form a dictionary
representation of the \textit{OSTIS Standard}.

\textit{\textbf{The OSTIS Glossary}}

:= \hspace{6mm} [Semantic electronic dictionary of Artificial intel-
ligence]

\end{multicols}
\end{document}

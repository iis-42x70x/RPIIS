\documentclass[10pt, letterpaper, twoside]{article}
\usepackage{multicol}
\usepackage{enumitem}
\usepackage[utf8]{inputenc}
\usepackage{fancyhdr}
\usepackage{graphicx}
\usepackage[paperwidth=230mm,paperheight=280mm]{geometry}
\usepackage{layout}
\usepackage{wrapfig}
\usepackage{enumitem}

\geometry{
 total={180mm,200mm},
 left=25mm,
 right=25mm,
 top=21mm,
 bottom=25mm,
 }
\begin{document}
\setcounter{page}{96}
\begin{multicols}{2}
In the context of the above-mentioned works, we
have formulated the task of developing a concept for
the construction of intelligent robotic systems based on
the use of knowledge, as well as outlined the basic
requirements for such systems:
\begin{itemize}[noitemsep]
   

\item  support of heterogeneous components of the robot
system, i. e. support of a single open interface of
interaction;

\item the possibility of transferring the knowledge accumulated
by the system during its operation to other
robotic systems with minimal changes;

\item adaptive design, i. e. the ability to change the composition
of the system components without having
to change the interaction logic;

\item support for self-modification of the system;
\item the ability of the system to expand and/or improve
its set of sensors and effectors;

\item the ability of the system to analyze the quality of
its physical and software components.

 \end{itemize}
These requirements correspond to the properties of
cybernetic systems given in the OSTIS technology standard
[6]. This technology is a reasonable choice for
designing an intelligent robotic system, as it ensures the
achievability of these properties.

In the process of building the concept of developing
systems of this type, it is necessary to form a list of
recommendations and general rules for designing intelligent
robotic systems, and to realize an applied intelligent
robotic system on the basis of the outlined theoretical
provisions.

\begin{center}
\MakeUppercase{\romannumeral3}. Proposed concept
\end{center}

The proposed robotic system design concept is developed
using OSTIS technology.

The fundamental possibility of integrating a physical
robotic system and OSTIS in the context of controlling
this system is based on the developed ontology that
includes a description of the basic physical components
of such a system (i. e., manipulators, transporters, etc.),
as well as the classes of actions that can be performed
by such components.

As a formal basis for knowledge representation within
the framework of OSTIS Technology, a unified semantic
network with a set-theoretic interpretation is used. This
representation model is called SC-code (Semantic Computer
code). The elements of the semantic network are
called sc-nodes and sc-connectors (sc-arc, sc-edges) [6].

Systems built on the basis of OSTIS Technology
are called ostis-systems. Any ostis-system consists of a
knowledge base, a problem solver and a user interface.
The basis of the knowledge base is a hierarchical system
of subject domains (SDs) and their corresponding
ontologies. Ontologies contain descriptions of concepts
necessary for formalization of knowledge within SD.
Any knowledge describing some problem, its context
and specification of solution methods can be represented
in the form of SC-code constructs. Thus, unification
of representation and consistency of different types of
knowledge describing problems, their context and solution
methods is ensured.

Benefits that can be achieved by using OSTIS as a
design tool for intelligent robotic systems include:
\begin{itemize}[noitemsep]
\item the possibility of isolating extracted knowledge,
independent of the manipulator types used, and
reapplying it to other robotic systems under development,
eliminating the need to code single-type
operations;

\item convenient means of visualization, for example, with
the use of SCg, which allows to determine the
working conditions of the components of robotic
systems;

\item use of open interaction interfaces that allow adding
other physical components on the fly;
\item explainability of the system operation modes, which
allows tracking the occurrence of emergency situations
with the formation of a detailed, humanunderstandable
report.
 \end{itemize}
\begin{center}

\MakeUppercase{\romannumeral4}. Robotics system design

\end{center}
 According to the proposed concept, we have carried
out the design and development of a collaborative robotic
intelligent system.

The main purpose of the developed system is to sort
objects of a certain type while maintaining the ability to
flexibly modify the filtering condition of objects. This
type of robotic systems is popular and widely used
in production conditions to select objects with certain
properties (for example, to reject manufactured products
or to organize them for subsequent packaging of only the
same type of goods). Such an operation, if automated, significantly
simplifies manual labor in production, reducing
the amount of monotonous work performed.

The physical part of the system consists of the following
components:
\begin{itemize}[noitemsep]
\item manipulators (2 units);

\item transporter;

\item single-board computer;

\item storage devices (general and for target objects);

\item tube;

\item camera;

\item ultrasonic sensor;

\item power supply;

\item indicator lamp;

\item auxiliary components such as conductors, relays,
voltage converters, and so on.
 \end{itemize}
The scheme of the main physical components of the
system is shown in Fig. 1.

The main components of the software part of the
proposed robotic system, as well as of any ostis-system,
are \textbf{knowledge base} and \textbf{problem solvers}. The user interface is the standard OSTIS technology tools for
viewing and editing knowledge bases. The computer
vision module, including a neural network model for
object detection, is also a program component.
\begin{center}
\begin{minipage}{.4\textwidth}
    \includegraphics[width=\linewidth]{пиво1.jpg}
    \scriptsize
    \begin{center}
        

    \caption{\textsl{Figure 1. The scheme of the main physical components of the system}}
        \end{center}
    \label{fig:enter-label1}
\end{minipage}
\end{center}
The functioning scenario of the developed system is
reduced to the following main actions:

1) picking up the object from the tube by the feeding
manipulator and moving it to the beginning of the
conveyor;

2) switching on the conveyor and moving the object
until the sensor is triggered;

3) disconnection of the conveyor at the moment of
sensor actuation;

4) recognition of the object (type and color) by means
of the installed camera;

5) moving the object from the conveyor belt to the
target object storage by means of the sorting arm,
if the recognized type and color match the set type
and color;

6) switching on the conveyor belt and moving the
object to the general storage;

7) switching off the transporter after the fulfillment of
item 6.

8) switching on the green signal of the indicator when
there are objects in the tube;
9) switching on the red indicator signal when there are
no objects in the tube.

Let’s describe the physical components of the system
in more detail.
\vspace{3mm}

\textsl{A. Physical components}
\vspace{3mm}

\textbf{Manipulators} are used for gripping and moving objects.
In this project, we used manipulators with different
types of grippers that are widely available on the market
— mechanical (pincer) grippers (Fig. 2) and vacuum
grippers (Fig. 3).

\textbf{Transporter} is intended for moving objects to the
specified point of technological operation (Fig. 4).

\textbf{Single} Board Computer — a specialized computer
on which the OSTIS platform is deployed and the logic
for controlling the system operation is implemented. In
our implementation, the SBC Raspberry PI 5 (Fig. \ref{fig:enter-label5}) [7]
was used for this purpose.

\begin{center}
\begin{minipage}{.4\textwidth}
    \includegraphics[width=\linewidth]{пиво2.jpg}
    \scriptsize
    \begin{center}
        

    \caption{\textsl{Figure 2. Manipulator with mechanical gripper}}
    
        \end{center}
    \label{fig:enter-label2}
\end{minipage}
\end{center}
\begin{center}
\begin{minipage}{.4\textwidth}
    \includegraphics[width=\linewidth]{пиво3.jpg}
    \scriptsize
    \begin{center}
        

    \caption{\textsl{Figure 3. Manipulator with vacuum gripper}}
        \end{center}
    \label{fig:enter-label3}
\end{minipage}
\end{center}

\textbf{Storages} — special containers used to store objects of
a certain type (Fig. 6).

\textbf{Tube} – a container for storing objects that is shaped
to be gripped by a manipulator (Fig. 7).

\textbf{Camera} is used to detect objects in the field of
view and determine their characteristics. We used a 2
megapixel FullHD backlit camera ZONE 51 LENS.

\textbf{Ultrasonic sensor} is used to identify situations in
which the object is in a given point of the conveyor. For our project we used the HC-SR04 sensor (Fig. 8) [8].
Power supply is required to power all physical devices
in the system. We used a 360-watt, 24-volt, 15-amp
power supply.

\textbf{Indicator lamp} is intended for light indication of the
\begin{center}
\begin{minipage}{.4\textwidth}
    \includegraphics[width=\linewidth]{пиво4.jpg}
    \scriptsize
    \begin{center}
        

    \caption{\textsl{Figure 4. Transporter part}}
        \end{center}
    \label{fig:enter-label4}
\end{minipage}
\end{center}
\begin{center}
\begin{minipage}{.4\textwidth}
    \includegraphics[width=\linewidth]{пиво5.jpg}
    \scriptsize
    \begin{center}
        

    \caption{\textsl{Figure 5. SBC Raspberry PI 5 with heat sink installed}}
        \end{center}
    \label{fig:enter-label5}
\end{minipage}
\end{center}
system status. In our project we used a TD-50 lamp with
two color options (red and green) (Fig. 9).
\vspace{3mm}

\textsl{B. Program components: knowledge base}
\vspace{3mm}

The ontological approach is used for knowledge structuring,
the essence of which is to represent the knowledge
base as a hierarchy of subject domains and their
corresponding ontologies. OSTIS technology provides a
basic set of ontologies on the basis of which ontologies
of applied ostis-systems are built.

The following subject domains are identified for the
considered intelligent robotic system:
\begin{center}
\begin{minipage}{.4\textwidth}
    \includegraphics[width=\linewidth]{пиво6.jpg}
    \scriptsize
    \begin{center}
        

    \caption{\textsl{Figure 6. Storage for objects with placed objects}}
        \end{center}
    \label{fig:enter-label6}
\end{minipage}
\end{center}
\begin{center}
\begin{minipage}{.4\textwidth}
    \includegraphics[width=\linewidth]{пиво7.jpg}
    \scriptsize
    \begin{center}
        

    \caption{\textsl{Figure 7. Fragment of the tube with placed objects}}
        \end{center}
    \label{fig:enter-label7}
\end{minipage}
\end{center}
\begin{center}
\begin{minipage}{.4\textwidth}
    \includegraphics[width=\linewidth]{пиво8.jpg}
    \scriptsize
    \begin{center}
        

    \caption{\textsl{Figure 8. Ultrasonic sensor}}
        \end{center}
    \label{fig:enter-label8}
\end{minipage}
\end{center}
\begin{center}
\begin{minipage}{.4\textwidth}
    \includegraphics[width=\linewidth]{пиво9.jpg}
    \scriptsize
    \begin{center}
        

    \caption{\textsl{Figure 9. TD-50 indicator lamp}}
        \end{center}
    \label{fig:enter-label9}
\end{minipage}
\end{center}
\begin{itemize}[noitemsep]
\item Subject domain and its corresponding robotic device
ontology;
 \end{itemize}



\end{multicols}
\end{document}
\documentclass[a4paper]{article}

\usepackage{multicol} %колонки
\usepackage{setspace} %межстрочный интервал
\usepackage{fancyhdr} %настройки верхнего и нижнего колонтитулов в документе.
\usepackage{newtxtext, newtxmath} % Задать шрифт Times New Roman
\usepackage{hyphenat}
\usepackage{scrextend}
\usepackage{enumitem}
\usepackage{mathtools}
\usepackage[left=2.5cm,right=2.5cm,top=2.5cm,bottom=2.5cm]{geometry}

\setlist[itemize]{noitemsep, topsep=0pt} % убрать отступы itemize
\fancyhf{} % очищает все верхние и нижние колонтитулы.
\renewcommand{\headrulewidth}{0pt} % remove the header rule
\cfoot{\textbf{\thepage}} % жирные номера страниц
\pagestyle{fancy}
\setcounter{page}{132} % настройка нумерации страниц
\setlength{\columnsep}{.5cm} % интервал между колонками

\begin{document}

\setlength\parindent{11pt}
\fontsize{9.7}{13}\selectfont

\begin{multicols}{2}


{\setlength{\parindent}{0pt}\nohyphens{\textit{F. Methods and technologies for solving the problem of collective activity organisation and consistency   }}} \par

\vspace{0.5mm}

{\fontsize{9.5}{13}\selectfont Today, some information is combined, structured and used in the form of common information repositories — Internet resources with their own databases or knowledge bases [13]. Databases and knowledge bases allow storing and processing information in the same place and automating the solution of information tasks of various kinds. That is, unlike non-digital analogues, information in databases or knowledge bases can be interpreted not only by a human but also by a computer .\par}

{\fontsize{9.5}{13}\selectfont This approach is the most promising in comparison with other existing approaches, as it allows solving some of the above mentioned problems. However, it is necessary to approach the solution of these problems in a methodologically correct way, in other words, to use, integrate and develop methods and technologies with the help of which it is possible to develop tools that allow solving all the problems listed above 
\par}


{\fontsize{9.5}{13}\selectfont  In order to create unified information resources with
the help of which specific activities could be easily made
consistent, it is necessary to apply modern technologies
and methods that enable \underline{effective} data and knowledge
management [14].}

{\setlength{\parindent}{0pt}\nohyphens{\textit{Knowledge-Based Engineering.}}} \par

{\fontsize{9.5}{13}\selectfont A popular area of research is \textbf{knowledge-based engineering}. Knowledge-based engineering (KBE) [15] is
a research area that explores methodologies and technologies for capturing and reusing knowledge in product
development. The goal of KBE is to reduce product development time and costs, which is primarily achieved by
automating repetitive design tasks while getting, storing,
and reusing knowledge about already designed products.}

{\fontsize{9.5}{13}\selectfont One of the hallmarks of the KBE approach is the automation of repetitive, non-creative design tasks. Automation not only provides significant time and cost savings,
but also frees up time for creativity, allowing more of
the design domain to be explored. This is facilitated
by another advantage of KBE: it allows knowledge to
be reused. As the researchers note, "about 20% of a
designer’s time is spent searching for and assimilating
information." This means that development information
and knowledge is not represented in a common and
easily accessible knowledge base. Obviously, in such
cases, reusing knowledge according to the established
KBE framework can save considerable time and effort.}


{\fontsize{9.5}{13}\selectfont The authors of the paper [15] highlight the challenges of using knowledge-based engineering and, first and foremost, emphasise the need to bridge the "technology gap" — lack of tools and technologies to support cost-effective KBE development and its application in the development of information systems for various purposes. According to the authors of the paper, such tools and technologies should provide search and reuse of knowledge, allow standardisation of this knowledge, describe the meaning of information and ensure transparency of information systems developed according to KBE principles.  
}

{\setlength{\parindent}{0pt}\nohyphens{\textit{Ontology-based approach.}}} \par

{\fontsize{9.5}{13}\selectfont Among the methodologies and technology of knowledge engineering, the description and structuring of knowledge by means of ontologies is most often given more attention [16]. The \textbf{ontology-based approach} is a methodology based on the use of ontologies to organise and represent knowledge about a particular subject domain [17]. In this context, "ontology" refers to the formal description of concepts, relationships, and attributes in a subject domain. 
}

{\fontsize{9.5}{13}\selectfont The widespread use of the ontological approach is explained by the fact that [18], [19], [20]: 
}

\begin{itemize}[leftmargin=5mm]
        
        \item Ontologies make it possible to provide \underline{consistency}
of concept systems between participants in some
process.
        \item Ontologies allow to \underline{organise} and \underline{classify} knowledge
from different disciplines, establishing links between
them, thus making knowledge more structured and
usable [21].
        \item Ontologies allow knowledge \underline{integration} by combining information from different sources and with
different representations, which helps to eliminate
semantic incompatibilities between different systems
and facilitates information sharing [22].
        \item Ontologies allow knowledge to be represented in a
form that computers can \underline{understand}. Thus systems
can automatically analyse and make decisions based
on the knowledge in ontologies
        \item  Ontologies provide a formal basis for developing \underline{expert} systems that can provide recommendations and advice in complex and multidimensional problems [23].

\end{itemize}

{\fontsize{9.5}{13}\selectfont The use of ontologies for interdisciplinary knowledge synthesis and integration helps in gathering and
analysing information from different domains and ensuring that it is properly understood and used in practice,
thus contributing to better knowledge utilisation and
more informed decision making.
}

{\fontsize{9.5}{13}\selectfont A series of works [24] emphasises the importance
of using ontologies to structure engineering knowledge.
According to the authors, ontologies reveal the semantics
of the information presented, eliminate heterogeneity in
the representation of multiple information sources, provide a common knowledge base for multi-agent systems,
provide semantics and structure for trust and reputation
systems, privacy-based systems, and codify common
knowledge across business and scientific domains. The
authors believe that the use of semantics as a central
mechanism will revolutionise the development and consumption of software and lead to the development of
software as a service — Software engineering 2.0.
}

\newpage \nohyphens{\fontsize{9.8}{13}\selectfont\setlength{\parindent}{0pt}According to the authors, an important challenge remains the problem of ensuring that knowledge bases are
consistent and coherent with each other, one solution to
which may be to develop a hierarchy of ontologies for
all systems in the form of a common knowledge base.
Knowledge bases help to simplify the storage of different
types of knowledge [23], and the ontology approach helps
to structure knowledge and the relationships between
them [19]. This provides a deeper understanding of the
information context and improves data availability and
integrity.} \par

{\fontsize{9.6}{13}\selectfont \nohyphens{The authors also highlight the importance of multiagent systems in ontology processing. A set of agents
in a multi-agent system can use this ontology as a
common knowledge base. This will greatly facilitate
communication and coordination between agents when
solving tasks together. The [24] describes the problem that methodologies for developing ontologies and
methodologies for developing multi-agent systems are
completely separate and have no connection with each
other. The authors believe that combining multi-agent
systems and ontologies for mutual use could revolutionise
information technology. \par}}


{\setlength{\parindent}{0pt}\nohyphens{\textit{Knowledge graph
}}} \par

{\fontsize{9.6}{13}\selectfont \nohyphens{Another popular method for describing complex structured knowledge and knowledge from different subject
domains is \textbf{the knowledge graph} [25], [26]. A knowledge
graph is a semantic network that stores information about
different entities and the relationships between them. An
entity or "node" of a graph can be anything: any material
object or abstract concept. Predicates or "edges" reflect
the relationships between different entities in the graph.
For example, Albert Einstein and the city of Ulm in
Germany are two separate entities, and the fact that
Einstein was born in Ulm is a predicate. \par}}

{\fontsize{9.6}{13}\selectfont \nohyphens{The use of graph models has the following advantages
[12], [27], [11], [13]: \par}}

\begin{itemize}[leftmargin=5mm]
        
        \item Graph data models have tremendous expressive
power. Graph databases offer a \underline{flexible} model of
data and a way to represent it. Graphs are additive,
providing the flexibility to add new data relationships, new nodes and new subgraphs to an existing
graph structure without compromising its integrity
and coherence.
        \item The diversity of data representation is minimized by
reducing the number of syntactic aspects, as graph
data models allow different types of knowledge to
be written in the \underline{same} way.
        \item To understand the meaning of knowledge, it
is necessary to represent this knowledge in an
\underline{understandable} form for everyone: both for a person
and for the system. Speaking about unification of
representation of all kinds of knowledge, it is considered important to use graph models not just as
means for storing structured data, but for storing semantically coherent and interconnected knowledge.
        \item \underline{Performance} of data processing is improved by one
or more orders of magnitude when representing
data in the form of graphs, which is explained by
the properties of the graph itself. Unlike relational
databases, where query performance degrades with
increasing query intensity as the dataset grows,
the performance of the graph data model remains
constant even as the dataset grows. This is because
data processing is localised in some part of the
graph. As a result, the execution time of each query
is only proportional to the size of the part of the
graph traversed to satisfy that query, not the size of
the entire graph.
        \item Graph models enable  \underline{efficient} semantic \underline{search}, i.e.
finding data and information based on the relationships between them, which helps to improve the
quality and accuracy of search queries, as well as
provides a deeper understanding of these relationships and dependencies between data
    
\end{itemize}

{\fontsize{9.6}{13}\selectfont \nohyphens{Knowledge graphs, like ontologies, help to link large
amounts of data from different sources into one common
knowledge collection. They can be general, i.e. storing
information about different types of data, and specialised,
focusing on a single subject domain. \par}}

{\fontsize{9.6}{13}\selectfont \nohyphens{Today, one of the largest knowledge graphs that stores
information from different subject domains is Wikidata.
Another good example of a knowledge graph is BioPortal
– the largest specialised graph with over 140 billion
facts about biotechnology and medicine. These graphs
are publicly available and accessible to all Internet users. \par}}

{\fontsize{9.6}{13}\selectfont \nohyphens{Systems that use knowledge graphs to represent and
process information are widespread. Google Knowledge
Graph — is a system used by the Google search engine
to provide structured data about user queries. Google
Knowledge Graph combines data from various sources
such as Wikipedia, Freebase and others to provide information about objects in various fields such as history,
culture, science and technology. \par}}

{\setlength{\parindent}{0pt}\nohyphens{\textit{Knowledge graph
}}} \par

{\fontsize{9.6}{13}\selectfont \nohyphens{Obviously, to ensure consistent and compatible activities, only technologies and tools for representing,
structuring, accumulating different kinds of knowledge
used in these activities are not enough. In addition, it is
necessary to create conditions and means to improve the
activity itself and its integration with other activities to
solve more problems. \par}}

{\fontsize{9.6}{13}\selectfont \nohyphens{One of such approaches that promotes the convergence of activities of specialists from different fields is
the methodology of collaborative design. Collaborative
design implies collective decision-making, open communication and active participation of all stakeholders
throughout the design process. By utilizing the power of
collaboration, this approach aims to create innovative and
effective solutions that meet the needs and desires of end \par}}

\newpage \nohyphens{\fontsize{9.5}{13}\selectfont\setlength{\parindent}{0pt} specially designed for those people who are beginning
to master the relevant field of knowledge, who have
not yet acquired the necessary qualifications. But it is
obvious that this implies a significant duplication of the
information presented.} \par

\vspace{1.0mm}

{\setlength{\parindent}{0pt}\nohyphens{\textit{E. Shortcomings of current solutions to ensure consistency and compatibility of different activities
}}} \par

\vspace{0.5mm}

{\fontsize{9.5}{13}\selectfont \nohyphens{All the tools considered for representing, structuring and accumulating information make it possible to
simplify the organisation and coordination of collective
activities, but do not allow to solve these tasks in a
comprehensive way, because: \par}}

\begin{itemize}[leftmargin=5.5mm]
        
        \item The increase in the number of reference materials
presenting and describing the same information in
different forms leads to an increase in duplication
and, consequently, inconsistency of this information.
        \item There is quite a lot of information in existing information resources that is characterised by inaccuracy,
unstructured, incomplete, incoherent and unreliable
information.
        \item Information becomes obsolete rather quickly, i.e. becomes irrelevant and unclaimed due to finding new
methods of solving existing problems. All irrelevant
information is quickly accumulated in the Internet
space. That is why there is a lot of so-called "junk
information" in Internet resources, which is this
irrelevant and unclaimed information.
        \item The input of information in information resources is
done by intermediaries - people who do not have the
necessary competence to modernise and disseminate
this information, which directly affects the quality of
all information. This is also due to the fact that a
person who receives information from one source
interprets and transmits it to another source in his
or her own way. Different people describe concepts
from different sources by synonymous terms, which
leads to the loss of the original meaning of these
concepts. Thus, new contradictions in information
appear.
        \item Working with huge amounts of information implies
working with several sources of this information. In
such sources it is difficult to search for necessary
(relevant) information, as there is a huge number
of different categories, which implies the use of
complex search operations.
        \item This, in turn, is related to the language of knowledge representation. The format of knowledge representation and description in reference materials is
understandable only to a human being and cannot
be processed by a computer system, and as a consequence, cannot be used for solving problems by a
computer.
        \item Knowledge is most often structured in the form of
books, encyclopaedias, dictionaries and reference
books on specific subject domains, which allows
one to learn a particular subject domain quickly.
However, this makes it difficult to understand information at the "junctions" of subject domains,
so that a person is not well-versed in interdisciplinary knowledge. The so-called "mosaicism" of
perception is formed in a human being, as a human
being during training and work gets used to artificial
division of knowledge areas and has difficulties in
solving problems at "junctions".

        \item To integrate information from different sources, algorithms for matching and merging data, identifying
and resolving duplicates, and algorithms for converting to common presentation formats are used, but
even these algorithms do not completely eliminate
inconsistencies and duplication of existing information.

        \item The existing information resources do not standardise and do not apply general principles of presenting
information for a wide range of readers. Each reader
perceives information in his/her own way, and consequently, there are differences in the understanding
of the same information.

        \item The popularisation of knowledge is carried out with
the help of specialised Internet resources that not
only simplify but also distort the presentation of information for professionally untrained readers. The
increase in the number of such Internet resources
contributes to the duplication of information and the
development of contradictions in it.
    
\end{itemize}

{\fontsize{9.5}{13}\selectfont \nohyphens{To solve these problems, methods and technologies
must be utilised which can: \par}}

\begin{itemize}[leftmargin=4.0mm]
        
        \item present any information in the same \underline{same} form;
        \item  \underline{integrate} integrate information from different information
sources;
        \item describe and  \underline{structure} structure information both from one
subject domain and information at "junctions" between subject domains;
        \item  \underline{standardise} standardise the description and visualisation of various types of information;

        \item  \underline{re-use} re-use existing knowledge and accumulate new
knowledge;

        \item present information in a form that is  \underline{understandable}
to both (!) humans and computers;

        \item develop tools to  \underline{quickly find} quickly find the information you
need;
        \item create a  \underline{personalised} personalised experience for any user;
        \item  \underline{develop} methods and tools to improve these methods
and technologies.
        
\end{itemize}

{\fontsize{9.5}{13.0}\selectfont \nohyphens{In other words, it is necessary to create such unified
integrated information resources, with the help of which
it is possible to quickly obtain existing information and
to integrate new information and it would be easy to
coordinate various activities, including activities on the
development of intelligent systems. It is also necessary \par}}

\end{multicols}

\end{document}

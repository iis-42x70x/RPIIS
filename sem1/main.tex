\documentclass [11pt, a4paper]{article}
\usepackage{graphicx} % Required for inserting images
\usepackage{multicol}

\usepackage{float}

\usepackage[english, russian]{babel}
\usepackage[utf8]{inputenc}
\usepackage[left=0.5cm,right=1cm,top=1.5cm,bottom=1.8cm]{geometry}
\usepackage{parskip}
\setcounter{page}{228}
\begin{document}
\setlength{\parskip}{0.05cm}
\begin{multicols}{2}

\begin{minipage}{.5\textwidth}
\includegraphics[width=\linewidth]{Снимок1.PNG}
\begin{center}
    \caption{\footnotesize{Figure 10. Task template}}
\end{center}

\end{minipage}
\begin{center}
\begin{minipage}{.5\textwidth}
\includegraphics[width=\linewidth]{Снимок4.PNG}

\begin{center}
    \caption{\footnotesize{Figure 11. Result of problem}}
    
\end{center}

\end{minipage}
\end{center}

\par
\\\quad The result is shown in the figure 11. First the \textit{cagent of task specification generation by template} created a specification of a task and then it called \textit{sc-agent of
solving a complex problem} which solved the problem because the \textit{knowledge base} had the statement that \textit{the tree is a connected acyclic graph}, the \textit{problem solver}
knows how to determine whether the graph is connected or not, whether the graph is acyclic or has cycle and
the \textit{problem solver} knows how to find the union of two graphs.

\begin{center}
     \subsubsection*{\textnormal{VII. Conclusion}}\addcontentsline{toc}{section}{VII. Conclusion}
\end{center}
  
\par\quad Discrete mathematics is applied in various fields including logistics, geographical information systems, computer science, modeling of physical and mathematical
phenomena, as well as sociology, biology, chemistry and economics, among others. Therefore, the development \textit {an intelligent system} to solve problems directly or indirectly related to discrete mathematics is of great relevance and importance in modern society. \par \quad Based on this work, the main components of \textit{intelligent tutoring systems for discrete mathematics} such as \textit{knowledge base, problem solver} and \textit{user interface} have been identified and described. In addition to this, the requirements that all the components of the \textit{intelligent tutoring systems} should follow and their functions were also identified. \par \quad Based on all the above, a prototype of \textit{intelligent tutoring systems for discrete mathematics} has been developed. However, this is only the beginning, and options for further development include the implementation of user tutoring and a personalized learning approach.
\\\quad
\begin{center}

    \subsubsection*{\textmd{\normalsize{Acknowledgment}}}
\end{center}
\par \quad The authors would like to thank the research groups of the Department of Intelligent Information Technologies of the Belarusian State University of Informatics and Radioelectronics
\\\quad
\begin{center}
    References
\end{center}
\begin{enumerate}
  \scriptsize{\item A. Liliya, “Social technologies and processes,” International
scientific journal "BULLETIN OF SCIENCE" No. 1, pp. 149–155,
2018.
\item (2024, March) Wolfram|Alpha. [Online]. Available: https:
//www.wolframalpha.com/
\item(2024, March) ALEKS - Assessment and LEarning in Knowledge
Spaces. [Online]. Available: https://www.aleks.com/
\item (2024, March) The WeBWorK Project. [Online]. Available:
https://openwebwork.org/
\item (2024, March) Maple T.A. [Online]. Available: https://www.
maplesoft.com/
\item Golenkov, N. Gulyakina, I. Davydenko, and D. Shunkevich,
“Semanticheskie tekhnologii proektirovaniya intellektual’nyh
sistem i semanticheskie associativnye komp’yutery [Semantic
technologies of intelligent systems design and semantic
associative computers],” Otkrytye semanticheskie tekhnologii
proektirovaniya intellektual’nykh system [Open semantic
technologies for intelligent systems], pp. 42–50, 2019, (In Russ.)
\item V. Golenkov and N. A. Gulyakina, “Graphodynamic models
of parallel knowledge processing: principles of construction,
implementation and design,” Otkrytye semanticheskie tekhnologii
proektirovaniya intellektual’nykh system [Open semantic
technologies for intelligent systems], pp. 23–52, 2012.
\item Bantsevich, “Structure of knowledge bases of next-generation
intelligent computer systems: a hierarchical system of subject
domains and their corresponding ontologies,” Otkrytye semanticheskie tekhnologii proektirovaniya intellektual’nykh system [Open
semantic technologies for intelligent systems], pp. 87–98, 2022.
\item Shunkevich, “Agentno-orientirovannye reshateli zadach
intellektual’nyh sistem [Agent-oriented models, method and
tools of compatible problem solvers development for intelligent
systems],” Otkrytye semanticheskie tekhnologii proektirovaniya
intellektual’nykh system [Open semantic technologies for
intelligent systems], pp. 119–132, 2018.
\item S. Sharapov, “Designing knowledge bases of intelligent
learning systems based on ostis technology,” A young scientist,
pp. 17–19, 2023. [Online]. Available: https://moluch.ru/archive/
478/105252/
\item G. Griswold, M. Shonle, K. Sullivan, Y. Song, N. Tewari,
Y. Cai, and H. Rajan, “Modular software design with crosscutting
interfaces,” IEEE software, vol. 23, no. 1, pp. 51–60, 2006.
}
\end{enumerate}
\begin{center}
\subsubsection*{ИНТЕЛЛЕКТУАЛЬНАЯ ОБУЧАЮЩАЯ
СИСТЕМА ПО ДИСКРЕТНОЙ
МАТЕМАТИКЕ}\addcontentsline{toc}{section}{ИНТЕЛЛЕКТУАЛЬНАЯ ОБУЧАЮЩАЯ
СИСТЕМА ПО ДИСКРЕТНОЙ
МАТЕМАТИКЕ}
\end{center}
\begin{center}
\large Шурмель К.А., Шарапов А.С., \par\quad \large Самохвал Е.С., Тищенко В.Н.
\end{center}
\par \quad В статье представлена модель интеллектуальной обучающей системы по дискретной математике. Модель такой системы использует методы и средства, рассчитанные на построение интеллектуальных обучающих систем по любой дисциплине и простую интеграцию новых дисциплин в существующую обучающую систему.

\begin{flushright}
\pushright{Received 01.04.2024}
\end{flushright}

\end{multicols}
\begin{center}

 \textbf{\huge{Methods and Means of Constructing Plans for
Solving Problems in Intelligent Systems on the
Example of an Intelligent System on Geometry}}
\\\quad
\begin{center}
   { Natallia Malinovskaya and Anna Makarenko\par
\textit{Belarusian State University of} \par
\textit{Informatics and Radioelectronics}\par
Minsk, Belarus\par
Email: natasha.malinovskaya.9843@gmail.com, anna.makarenko1517@gmail.com}
\end{center}

\end{center}
\begin{multicols}{2} 
\par\quad\textbf{ \textit{Abstract}—In this paper we propose an approach to the
development of methods and tools for constructing plans for
solving problems in intelligent systems on the example of
an intelligent system for geometry. The described approach
is aimed at improving the accuracy of answers due to
the possibility of decomposition of problems into simpler
ones, and also aims to overcome the shortcomings of
modern intelligent systems. An intelligent system realizing
the proposed approach is described.}
\par\quad\textbf{\textit{Keywords}—problem solving, knowledge, knowledge base,
intelligent systems.}
\begin{center}
    

\subsubsection*{\textnormal{I. Introduction}}
\end{center}
\par \quad Nowadays, the use of intelligent systems in various
fields is becoming relevant. However, the quality of ISs
is largely determined by its answers and the ability to
solve complex problems, which is why it is necessary that
its answers are as accurate and reliable as possible. In
order to increase the accuracy of answers it is necessary
to be able to decompose problems into simpler ones, in
turn, the automation of this process will allow the IS to
solve not only simple problems, but also complex, nontrivial ones. The relevance of systems that have the ability not only to perform the above functions, but also have
the ability to partially satisfy the human need for quick
answers, allowing you to automate some of the routine
actions.\par \quad Problem solver is one of the key components of
ISs, allowing them to solve a wide range of problems.
Unlike other modern software systems, the peculiarity of
problem solvers in ISs is the necessity to solve problems
in conditions when the necessary information for their
solution is not explicitly localized in the knowledge base
of the IS and must be found in the process of problem
solving on the basis of certain criteria. \par \quad The composition of the problem solver in each particular system depends on its target, classes of solved
problems, subject area and other factors. In general, a
problem solver provides the ability to solve problems
related both to the core functionality of the system
and to ensure its efficient operation and development
automation. A problem solver that performs all these
functions is called a unified problem solver for a given
IS. \par \quad Expanding the areas of application of ISs requires
their ability to solve complex problems, which involve
the joint use of different models of knowledge representation and problem solving models. In addition, solving complex problems involves the use of shared information
resources, such as a knowledge base, by different components of the solver that specialize in solving different subproblems. Since a complex problem solver integrates
different problem solving models, it is called a hybrid
problem solver. \par \quad Examples of complex problems are:
\begin{itemize} 
 \item problems related to understanding natural language
texts (both printed and handwritten), understanding
speech messages and images. In each of these cases
it is required to perform syntactic analysis of the
processed file or signal, remove insignificant fragments, classify significant fragments, relate them to
concepts known to the system, etc.;
 \item  automation of adaptive learning for schoolchildren
and students, which implies that the system is capable of autonomously solving various problems
from a certain subject area, as well as managing
the learning process, generating tasks for students
and controlling their independent fulfillment by the
student;
 \item the problems of planning the behavior of intelligent robots, which involve both understanding a
variety of external information and making various
decisions using both credible methods and methods
that rely on probabilistic estimates and plausible
assumptions;
 \item problems related to complex and flexible automation
of various enterprises.
 \item etc.

\end{itemize}
\par \quad The use of different problem solving models within
an IS implies decomposition of a complex problem into
229
subproblems that can be solved using one of the known
IS problem solving models. Due to the combination of
different problem solving models, the set of problems
solved by the hybrid solver will be much wider than the
combination of sets of problems solved separately by all
problem solvers included in its composition [1].
\par \quad The existing variety of approaches to problem solving
in computer systems can be divided into two classes:
\begin{itemize}
    \item The use of different problem solving models within
an IS implies decomposition of a complex problem into
229 subproblems that can be solved using one of the known
IS problem solving models. Due to the combination of
different problem solving models, the set of problems
solved by the hybrid solver will be much wider than the
combination of sets of problems solved separately by all
problem solvers included in its composition [1]. 
The existing variety of approaches to problem solving
in computer systems can be divided into two classes:
\begin{itemize}
    \item[-] programs written in both imperative and declarative programming languages, including logical
and functional programming [2];
 \item[-] genetic algorithm implementations [3], [4];
 \item[-] neural network models of knowledge processing
[5], [6], [7].

 
\end{itemize}
It should be noted that even in the case of using a
stored program, the solution of the problem is not2
always trivial, because, first, it is required to find
such a stored program on the basis of some specification, and second, to provide its interpretation;
\item solving problems in conditions where the solution
program is not known.
\end{itemize}
In this case, it is assumed that the system does not
necessarily contain a ready-made solution program for
the class of problems to which some formulated problem
to be solved belongs. In this connection, it is necessary to
apply additional methods of searching for ways to solve
the problem, which are not designed for any narrow class
of problems (e.g., splitting the problem into subproblems,
methods of searching for solutions in depth and width,
method of random solution search and trial-and-error
method, etc.), as well as various models of logical
inference (classical deductive, [8], inductive [9], [10],
abductive [8]; models based on fuzzy logics [11], [12],
[13], the logic of default [14], temporal logic [15], and
many others).
\par \quad For example, currently one of the most popular approaches to text generation for natural language problems is the use of \textit{large language models} [16], which are models consisting of neural networks with many parameters trained on a large amount of unlabeled text, but these
models have a number of disadvantages, partial solution
of which can be prevented by integration of such systems
with knowledge bases [17], but this approach is not able
to solve all the disadvantages of large language models
in solving complex problems.
\par \quad Thus, although there have been significant advances
in the development of problem solvers for ISs, there are
still outstanding challenges related to provisioning:
\begin{itemize}
  \item compatibility of different private problem solvers,
i.e. the possibility of their coordinated use in solving
the same complex problem;
\item possibilities to modify the hybrid solver without significant additional costs in the process of operation
of the IS. This includes expanding the number of
used problem solving models without restrictions
on their type. This requirement is due to the fact
that when solving a complex problem, it may be
unknown what specific problem-solving models and
types of knowledge will be needed.

\end{itemize}
\par \quad The modern development of \textit{artificial intelligence} is
moving towards the creation of intelligent computer
systems of a new generation [18]. These systems are
capable not only of solving problems from various fields
of knowledge, but also of explaining their solutions.
However, the disadvantages described above do not allow such systems to be built solely on the basis of
existing solutions. Instead, new-generation ISs are built
on a unified knowledge base that integrates problems,
subject domains, and methods of their solution. Thus,
we conclude that the use of modern solutions such
as neural networks, models using specialized software
interfaces between different system components, large
language models can and should become a powerful
tool for solving problems of ISs [19], but they cannot
completely replace these systems [20].
\par \quad That is, despite the fact that currently there is a large
number of problem-solving models, many of which are
implemented and successfully used in practice in various
systems, the problem of low consistency of the principles
underlying the implementation of such models and the
lack of a single unified framework for the implementation
and integration of different models remains relevant,
which leads to the fact that:
\begin{itemize}
    \item it is difficult to simultaneously use different models
of problem solving within one system when solving
the same complex problem;
    \item it is practically impossible to use technical solutions
realized in one system in other systems;
    \item in fact, there are no complex methods and tools
for building problem solvers, which would provide
the possibility of designing, realizing and debugging
solvers of different kinds.
\end{itemize}
\par \quad Therefore, the ability of an IS to independently solve
non-trivial problems will expand the possible functionality of the IS, detail and improve the accuracy of its
answers.
\par \quad The purpose of this study is to refine the ontologies of
actions and problems, to develop a collective of agents
that allows to divide the problem into subproblems, as
well as to develop a methodology for the application of
the obtained subsystem in specific application systems
on the example of an IS for geometry.
\end{multicols}
\end{document}

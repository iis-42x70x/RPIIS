\documentclass[10pt, a4paper]{article}
\usepackage{graphicx}  %вставка изображений
\usepackage[utf8]{inputenc}  %кодировка
\usepackage{import}
\usepackage{graphics}
\usepackage{float}
\usepackage{scn}

\usepackage{multicol}  %столбцы
\usepackage{microtype}  %типография
\usepackage{enumitem}  %настройка списков
\usepackage{setspace}  %межстрочный интервал
\usepackage[english,russian]{babel}
\usepackage[left=1.8cm,right=1.9cm,top=2.2cm,bottom=2.5cm,]{geometry}

\justifying
\fancyhf{}
\setcounter{page}{111}
\renewcommand{\thepage}{\textbf{\arabic{page}}}
\footskip=5em
\singlespacing
\makeatletter
\newcommand*{\rom}[1]{\expandafter\@slowromancap\romannumeral#1@}

\begin{document}

\begin{multicols}{2}


\noindent\normalsize the history of their purchases, views, and liked goods. The
success of the marketplace forced competitive companies
to pay attention to the recommender systems technology. In
the rapidly growing segment of social networks business\-oriented network LinkedIn also uses recommender system.
Its system offers a member of the community the closest
to the interests, specialization and experience of the
community, company, specialists. To build recommender
systems, there are four main types of data filtering [12]:

\vspace{-0.9em}
\begin{itemize}[itemsep=0pt, parsep=0pt]
\setstretch{1.06}
\item collaborative filtering;
\item content filtering;
\item knowledge-based filtering;
\item hybrid filtering.
\end{itemize}
\vspace{-0.9em}

 Often the recommender system technology is based on
the principles of collaborative filtering, which analyses
the actions of the most similar users with a similar profile.
However, other types of filtering are also used in practice.

\vspace{-0.5em}
\begin{flushleft}
 \textit{A. Collaborative filtering}
 \end{flushleft}
 \vspace{-0.9em}

\hspace{-1.3em}
The main principle of the functioning of filtering
programs is the assumption that users with the same
interests will subsequently have similar preferences. For
the effective functioning of the model, not only the
previous behavior of the client, its query history, but also
the corresponding parameters of the additional cluster
of similar users are taken into account. The target of
collaborative filtering is to identify a certain number of
customers operating with the closest patterns of behavior,
and to recommend in further the goods or services liked
by this group.\par
The next method, which in turn is already based on
comparing the similarity of objects, is called item-based.
Its principle can be formulated as follows: if users who
rated two products liked both, then the following users
who tried only one product can be offered another.

\vspace{-0.3em}
\begin{flushleft}
 \textit{B. Content filtering}
 \end{flushleft}
 \vspace{-0.9em}

\hspace{-0.8em}
Content filtering is based on the assumption of the
constancy of user interests. In other words, based on the
client’s past activity, it can be argued that in the future
he will be interested in similar objects. Content filtering
uses the following input data: both a set of users and a set
of categories that correspond to users’ queries and to the
objects, which users like.\par
\hspace{-0.8em}
\fontsize{9.6}{12}\selectfont {The purpose of this type of recommender systems is
to build such a variety of items that are closest to the
favorite categories of the current user. The main task of this
methodology is to search for objects potentially close to
the interests of users among a set of objects not yet viewed
by the client. This search is based on finding the similarity
of objects with the user’s interests known to the system.
The absence of the need to revealing large user groups to
ensure the functionality of the method is one of the main
advantages of content filtering. This method also avoids the
problem of “cold start”, because each object has attributes}

\columnbreak

\normalsize\noindent that will be analyzed in future. This type of filtering often
make combination with collaborative filtering.

\vspace{-0.7em}
\begin{flushleft}
\textit{C. Knowledge-based filtering}
\end{flushleft}
\vspace{-1em}

\hspace{-0.8em}
The most resource-intensive approach is to develop
knowledge-based recommender systems. The main source
of information is not the user assessment of objects or
their metadata, but the rules and conditions developed by
experts and expert systems for forming recommendations.
Some researchers consider content filtering as a special
case of the knowledge-based filtering, however, due to the
wide prevalence, most prefer to classify it as a separate
type. For the functioning of this kind of recommender
systems, it is necessary to distinguish many expert rules,
similarity metrics and user interest objects. For practical
application of a given rules set, it is necessary to define
user’s interests and preferences in terms of the subject area.\par
The knowledge-based approach requires deep
understanding of the technical features of the product,
creation of user scenarios, inclusive restrictions and
rules. Undoubtedly, the use of knowledge-based systems
improves the quality of the recommendations being
formed, since user requests find the most accurate
response from recommendation algorithms among
all methodologies. In addition, this method will be
indispensable in those areas of commerce where the
number of regular customers is relatively small. Of
course, the development of such systems is extremely
time-consuming in terms of time and resources. To
improve the accuracy of functioning, it is necessary to
involve relevant specialists in the field of data collection
and processing, building the necessary models and user
behavior. Also, such systems require additional interaction
from the client with the system, which can lead to the
outflow of some part of the target audience, moreover, the
collected data cannot always be correctly interpreted by
software.

\vspace{-0.5em}
\begin{flushleft}
\textit{D. Hybrid filtering}
\end{flushleft}
\vspace{-0.9em}

\hspace{-1.3em}
The last type of recommender systems, hybrid,
as the name suggests, is a synthesis of two or
three above-mentioned methodologies. The use of
hybrid recommender systems increases the efficiency,
performance and accuracy of algorithms, and compensate
their lacks. For example, the most used combinations are:

\vspace{-1.2em}
\begin{itemize}[itemsep=0pt, parsep=0pt]
\setstretch{1.06}
\item combining collaborative and content filtering approaches (with different weights);
\item using some content-based filtering properties in
collaborative filtering algorithms;
\item partial using of knowledge-based filtering rules in recommender systems based on content filtering;
\item building a separate model corresponding to business needs and subject area terms, combining rules of all three types.
\end{itemize}
\vspace{-1em}

There is no unified algorithm for the functioning of
hybrid systems, which allows researchers to apply a wide

\columnbreak

\noindent range of modern methodologies to create unique models. It
was the hybrid type that became the basis of recommender
systems in large companies to ensure better personalized
interaction of users with their services.\par
\hspace{-0.8em}
Let’s take a closer look at two of the most popular
methods of recommender systems – collaborative filtering
and content filtering. Algorithms of these types of filtering
can be classified into three main categories:

\vspace{-0.95em}
\begin{itemize}[itemsep=0pt, parsep=0pt]
\setstretch{1}
\item anamnestic methods, or methods based on the
analysis of available estimates (memory-based
filtering), are a family of algorithms that are based
on statistical methods, the purpose of which is to
search for the nearest group of users to the analyzed
user; this approach is similar to the closest neighbors
method, and recommendations are formed as a result
of calculating a similarity measure based on a matrix
of estimates of the users in the database; the main
representative of memory-based algorithms is the
used-based and item-based weighting of estimations;

\item model-based filtering, in which a descriptive model of user and object assessments is preliminary
formed, and priority relationships between them
are distinguished; the main complexity of this
method is its preliminary stage, where resourceintensive training of the model takes place; different
approaches can be used to building such a
model: cluster analysis methods, Markov decision
process (MDP), singular value decomposition (SVD),
latent semantic analysis (LSA), principle component
analysis (PCA), etc.;

\item hybrid methods, which suppose synthesis of several approaches to achieve a better result; for example,
collaborative filtering systems can take advantage
of a relatively easy-to-interpret anamnestic method
with the efficiency and performance of model-based
methods, the purpose of which is to increase the speed
of recommender system work.

\end{itemize}

\vspace{-0,8em}
Problems of development of recommender systems may
include following situations:
\vspace{-1em}

\begin{itemize}[itemsep=0pt, parsep=0pt]
\setstretch{1.06}

\item sparseness of data, which means that due to a huge number of data the matrix “object-user” in system’s database becomes difficult to processing that
complicates overall algorithm’s work;

\item scalability, which means that traditional data
processing algorithms may not cope with the growing
flow of new customers and the goods they evaluate;
for example, it can be extremely difficult to perform
operations on matrices illustrating information about
tens of millions of users and hundreds of thousands
of objects, especially because the requirement for
modern recommender system is an instant response
to customer requests from all over the world;

\item “cold start”, which arises in the case of new
customers and goods emergence, because the
absence of the information about the previous user

\columnbreak

\noindent interaction; this problem can be partially solved by
the use of content/knowledge analysis or so-called
"average"user;

\item the lack of unified names of analyzed objects
(especially in users’ queries) may have negative
influence on the efficiency of joint filtering methods;
recommender systems do not have the ability to define
a hidden speech association, which can lead to the
including of the same objects to different classes; for
example, an algorithm will not be able to find the
coincidence of the “toys for children” and “children’s
toys” queries;

\item fraud, for example, companies interested in the
profitable sale of their own products can artificially
underestimate the goods of their competitors and
wind up a positive rating of their products, that will
lead to the recommendation of the products of firms
using such frauds;


\item market diversity, which allows users to explore the
vast expanses of marketplaces in better products
search, and such consumer boom does not always
correlate with the work of some methods of
collaborative filtering, for example, purely based on
the rating and success of sales of goods that do not
take into account the possibility of promoting little\-known, new goods, which can adversely affect the
diversity of the market and will lead to the survival
of only the largest market players to whom the main
user attention has already been focused;

\item presence of the clients at the market, whose opinion is
strikingly different from the majority; for such users,
algorithms may not find unique like-minded people
suitable for users, it can reduce the quality of their
personalized recommendations.

\end{itemize}

\begin{center}

\vspace{-1em}
\rom{5}. Experience of DSS and recommender systems
engineering

\end{center}

\vspace{-1.5em}
There are several main results in DSS engineering have
been received by the authors:
\vspace{-2em}


\begin{itemize}[itemsep=0pt, parsep=0pt]
\setstretch{1.05}

\item methods of efficiency assessment of the rule and
model bases in DSS have been elaborated, which
include the following coefficients: rule base certainty,
rule base coverage, rating class efficiency and
rating efficiency; formulae for calculation of these
coefficients are deducted with the using of the rough
sets theory [13];

\item theoretical and practical approaches of DSS
engineering for stock markets have been developed,
which include the technology of the liquidity
evaluation [14], the technology of algorithmic trading
by means of 5-component oscillator [15], the
technology of securities prices prediction with the
use of the neural network [16];

\item the concept of algorithmic marketing and machine
learning in relation to the marketing activity, which

\columnbreak

\noindent is based on decision trees and ABC-analysis and
allows reducing time necessary for market big data
processing [17];

    
    \item the methodology of multi-agent DSS design
and developing, which includes approach to the
modelling of representation of knowledge about the
subject area on the base of the concept of generalized
objects [18];

    \item adaptation of the technology of DSS engineering
to recommender systems design, which will be
considered below.
    
\end{itemize}
\vspace{-0.8em}

Example of the recommender system was realized
for the purpose of the choice of musical tracks for
different sport training. The initial data for analysis are
the most popular audio sets for a specific training session
(playlists of reputable sports publications, well-known
fitness instructors and trainers, popular music editions).\par
For the study 4 types of sports training were used:

\vspace{-0.8em}
\begin{itemize}[itemsep=0pt, parsep=0pt]
\setstretch{1}

    \item yoga, aimed at achieving internal harmony and
tranquility;

    \item cardio training, including a set of intensive exercises
that increases the heart rate;

    \item running, aimed at increasing endurance;

    \item power exercises that contribute to an increase in
muscle strength.
    
\end{itemize}
\vspace{-0.8em}

Client-server representational state transfer architecture
(REST) is used for the integration of the recommender
system with the mobile applications [19]. General scheme
of interaction between Android client and database in
REST architecture is represented in Fig. \ref{fig1}. The main
requirements to this client-server architecture are the
following: the server cannot store client information
between requests, that’s why interface should be unified.

\vspace{0.5em}
\begin{figure}[H]
\centering
\includegraphics[width=0.5\textwidth]{mat_1.png}
\caption{\footnotesize  General scheme of interaction between Android client and
database in REST Architecture.}
\label{fig1}
\end{figure}

\vspace{-0.5em}

\normalsize Due to creation of music track pattern for each sport
training it is necessary to create a big data set on the base
of expert data. A large number of listeners (at least 15,000
people) will serve as an indicator of quality. As a result, a
set of 20 playlists was formed for each class, totaling more
than 1000 unique tracks, on the base of which the desired
average was calculated. Then, unique playlist identifiers
are read from the generated files in order to eventually
process each composition from the audio selection and
\columnbreak

\noindent sequentially extract all the properties of the tracks from it. To form the desired patterns, the function of calculating the average value of the set of components was used.\par
As a result, 10 average indices represented in
Fig. \ref{fig2} have been calculated: acousticness, valence,
danceability, energy, instrumentalness, key, liveness,
loudness, speechiness, tempo.

\vspace{-0.5em}
\begin{figure}[H]
\centering
\includegraphics[width=0.5\textwidth]{mat_2.png}
\caption{\footnotesize Example of classification of the musical tracks by recommender
system.}
\label{fig2}
\end{figure}
\vspace{-0.8em}

\normalsize After forming a pattern sample for each of the classes,
it is necessary to select a set of tracks that are most similar
to the pattern (200 songs are selected in the application).
The search for songs is carried out from a large external
data set, the storage of which is organized in a .csv file.\par
For example, the program uses a file with 32880 unique
compositions, where, in addition to a unique identifier,
genre, the name and listing of the artists participating in
the recording of the song, the extracted audio properties
are presented. The playlist downloaded as a result of the
operation of the HTTP library is displayed as interactive
buttons, clicking on which leads to the playback of the
corresponding song (Fig. \ref{fig3}).

\vspace{-0.5em}
\begin{figure}[H]
\centering
\includegraphics[width=0.5\textwidth]{mat_3.png}
\vspace{-2em}
\caption{\footnotesize Interface of the mobile recommender system for choosing musical
tracks for different sport trainings.}
\label{fig3}
\end{figure}
\vspace{-0.7em}

\normalsize The recommender system also has own media player,
where it is possible to pause/continue playing a song, turn
on the next/previous composition; and mix and loop modes
are available.

\end{multicols}

\end{document}



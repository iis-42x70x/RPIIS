\documentclass[a4paper]{article}

\usepackage{multicol} %колонки
\usepackage{setspace} %межстрочный интервал
\usepackage{fancyhdr} %настройки верхнего и нижнего колонтитулов в документе.
\usepackage{newtxtext, newtxmath} % Задать шрифт Times New Roman
\usepackage{scrextend}
\usepackage{enumitem}
\usepackage[left=2.5cm,right=2.5cm,top=2.5cm,bottom=2.5cm]{geometry}
\setlist[itemize]{noitemsep, topsep=0pt} % убрать отступы itemize
\fancyhf{} % очищает все верхние и нижние колонтитулы.
\renewcommand{\headrulewidth}{0pt} % remove the header rule
\cfoot{\textbf{\thepage}} % жирные номера страниц
\pagestyle{fancy}
\setcounter{page}{196} % настройка нумерации страниц
\setlength{\columnsep}{.5cm} % интервал между колонками

\begin{document}

\setlength\parindent{10pt}
\fontsize{10}{13}\selectfont

\begin{multicols}{2}

{\setlength{\parindent}{0pt} % Убрать в самом начале отступ
 {
with objects on the ground and time intervals provides
opportunities to determine cause-and-effect relationships,
to identify groups of similar data, and to predict future
events.}}

{\fontsize{9.92}{13}\selectfont On the one hand, the representation, integration and
processing of spatially referenced data is the task of a
corresponding class of systems called geographic information systems (GIS). On the other hand, the focus on
using spatially referenced data to establish semantic links
between spatially referenced data and the knowledge of
the subject areas for which the GIS is being developed indicates the need to use artificial intelligence technologies
and design intelligent systems.
\par


It is noted in the work [6] that at the current level
of development geographic information systems have
become practically the main tool for modeling natural,
economic, social processes and situations, tracing their
relationships, interactions, predicting further development in space and time, and most importantly a means
of providing (supporting) decision-making management.
Modeling in geographic information systems is based
on databases and knowledge bases. The former integrate
digital cartographic, aerospace, statistical and other data
reflecting the spatial position, state and attitude of objects,
and the latter contain sets of logical rules, information,
concepts necessary for modeling and decision-making.
At the same time, GIS is a special technology based on
computer complexes and software tools.\par

 In the past decade, remote sensing data have been
the main source of new data about the Earth, which
necessitates the creation of an information system with
specialized services that allow scientists and specialists to perform thematic processing of remotely sensed
data by changing the data processing parameters in a
certain way and to analyse the obtained information
independently [4]. At the same time, large crowdsourced
geographic datasets have been generated about the Earth
today as a result of the observed web phenomenon
known as Volunteered Geographic Information (VGI) [7]
through the development of spatial information systems
and web mapping projects, the main ones being:\par


\begin{itemize}[leftmargin=5mm]
        
        \item Yandex search and information mapping service [2];
        \item a non-commercial web-mapping project to create
a detailed free and free geographical map of the
world OpenStreetMap (OSM) by the community of
participants — Internet users [1];
        \item Google Maps is a set of applications built on top of
the mapping service provided by Google [3].
    
\end{itemize}



 The [8] argues that the growth of web services and applications for geographic information systems has made
large archives of spatial data available over the Internet.
Significant advances in GIS web service development
technologies have resulted in several examples of mapping and graphics services that conform to web service
standards and provide geospatial data and digital maps to
enterprise developers. Thus, both government surveying
and mapping services and private sector enterprises have
recently experienced a surge in the development of web
services and web-based applications for GIS, making
large archives of spatial data available over the Internet \par


 In this regard, the role of the map as an image-sign
geoinformation model of reality for quick and adequate
perception of information is acquired. Creation of maps
in electronic form, using GIS–technologies, is the most
important task of modern society, because it is the map
that becomes the tool with which a person can make a
decision, from the simplest to the most complex, even
in emergency situations. Accordingly, the society makes
more and more demands to maps, the user, referring to
the map, wants to receive reliable information and from
a huge array of data to choose only the information that
would be more suitable for making the right decision [9]. \par


 In addition, new and more sophisticated data collection
technologies (knowledge bases based on wiki technologies, classifiers, natural language parsing, etc.) are now
available. The large amount of accumulated geospatial
data generated by Earth observation satellites as well
as ground-based devices and sensors offers enormous
potential to address global social issues related to natural
disasters, health, transportation, energy and food security
[10], [11]. Interoperability is particularly important as
the level of cooperation between information sources
at national, regional and local levels increases, requiring new methods to develop interoperable geographic
systems [12]. Therefore, the use of terrain objects as
integrating elements in information systems is essentially
interdisciplinary in nature, as they integrate research in
economics, ecology, climate forecasting, terrain development, formation of optimal routes, and more.
\par


 In the industry of geoinformation systems development
nowadays there is a need in their intellectualization, i. e.
in solving problems traditionally related to geoinformatics with the use of artificial intelligence methods. First
of all, these are the tasks of intelligent search. Existing
instrumental GIS, which are the means of development
of applied GIS, do not solve the problems of intelligent
search for a number of reasons, among which we will
emphasize the following:
\par

\begin{itemize}[leftmargin=5mm]
        
        \item practically all of them are based on internal (closed)
formats of spatial data representation, and exchange
open formats serve only as a means of map data
exchange between different GIS tools;
        \item thematic data are mapped to specific spatial objects and exclude the possibility of establishing links and
relationships between such data;
\item  implementation of applied tasks of geoinformatics
is carried out in internal programming languages,
thus only simplifying access to spatial data, and the
map serves as a means of visualization.
\end{itemize}

 
 In the field of GIS development it is necessary to
emphasize the problem of formation of cartographic
images from information resources, for the solution of
which the methods of dynamic representation of spatial
data in GIS [13] are proposed, as well as the unsolved
to date problem of information integration.
\par

Thus, a group of international geographic and environmental scientists from government, industry, and
academia brought together by the Vespucci Initiative for
the Advancement of Geographic Information Science,
and the Joint Research Centre of the European Commission [14], argue that despite significant progress, the
ability to integrate geographic information from multiple
sources is very limited and in order to facilitate such
integration, an understanding of the statistical challenges
of integration at different scales is needed, as well as the
study of linguistic services\par
A mathematical model is proposed to facilitate the
integration of spatial information and attribute data,
which enabled the researcher to reduce the time to obtain data for management decision making in municipal
services [15].\par
It should be noted that the need for information integration requires semantic geo-interoperability and harmonized understanding of the semantics of geodata [16].
Interoperability is an indicator of effective communication between systems [17]–[19].\par
On the other hand, known technologies of designing
intelligent systems use cartographic materials, as a rule,
in the form of raster images, i. e. there is no possibility
to consider a map as a set of geographical objects with
specified topological and subject-oriented (depending on
the type of map) relations, while it is argued in the
paper [9] that we need new maps, the content of which
is supplemented with spatial knowledge, corresponding
to the subject area for the preparation of spatial maps.\par
Besides, for a fixed territory the same objects of
terrain and phenomena are used in different application
areas: epidemiology, construction, environmental protection and nature management, land relations, etc., which
determines the necessity to harmonize the ontology of
subject areas with the objects of terrain and phenomena
inherent in a given territory, thus providing a vertical
(subject-oriented) level of GIS design.\par
Note that when designing a GIS for a new territory,
the basic functional requirements are preserved and it
is necessary to take into account not only the previous
experience of GIS design, but also to use previously
designed functional components, i. e. we are talking about the horizontal level of GIS design, when the
territorial area is expanded and systems are designed for
new territories.\par

The third aspect is the temporal component, relevant
for retrospective analysis and modeling, thus providing
a dynamic GIS that can deal with terrain objects and
phenomena within a specific time period.\par
Currently proposed GIS tools have weak compatibility due to the lack of unification of subject knowledge
with ontologies of terrain objects and phenomena, as well
as with temporal components.\par
Known research on the integration of spatial data and
domain knowledge to ensure semantic interoperability
has been conducted for systems based on the Semantic
Web technology stack RDF, RDFS, OWL and the Web
Ontology Language OWL provides advanced capabilities
for describing the subject areas of interacting systems
and provides machine-interpretable definitions of fundamental concepts in the subject area and the relationships
between such concepts in the ontology.\par
Recently, due to the development of Semantic Web
technology, the key element of which is ontologies, it
has become possible in GIS to emphasize the semantics
of subject knowledge, to integrate and merge different
datasets in related fields, to establish subject rules and
their recording using RDF (Resource Description Framework) [20]–[22]. This capability certainly enhances the
capabilities of GIS technologies. However, in order to
do so, several important tasks must be solved. These
are, first, justifying the use of tools to integrate spatial
data and subject knowledge [23], and second, computing the similarity between geospatial objects that belong to
different data sources [24]–[26].\par
For example, the paper [27] states that there are
research problems related to the integration of different
types of geographic information. The authors propose
to base the GIS architecture on ontologies acting as a
system integrator in order to ensure smooth and flexible integration of geographic information based on its
semantic value. In this approach, the ontology system is
a component, such as a database or knowledge base (in
general case, an information component), interacting to
achieve the goals of the geographic information system,
and viewing the ontology, allows the user to obtain
information about the existing (formalized) knowledge
in the system. The use of several ontologies eventually
allows to extract information at different stages of classification, i. e. for different types of information used for
the purposes and in the interests of GIS. These ideas are
developed in the works of [28]–[30].\par
The process of ontology development is called ontology engineering and according to the concept of ontology
engineering, ontologies must be developed before they
can be used in a GIS. Thus, a GIS is based on a


 subject area described initially by an ontology model, with ontologies acting as a tool for knowledge generation
[31]. \par
{At present, scientific areas are developing so-called
Smart-systems aimed at qualitative improvement of technical and economic indicators within the subject area.
The application of geoinformation technologies for scientific research in the subject areas in conjunction with
traditional tools, methods and models of artificial intelligence allow obtaining qualitatively new scientific results,
as well as aimed at reducing the time of searching
for acceptable solutions for the set tasks. At the same
time, the authors pay special attention to the integration
of terrain objects and data and knowledge in system
research of a particular subject area.\par
Thus, Massel L. V. et al. proposed a methodical
approach to the integration of remotely sensed earth
observation data based on the methods of data and
knowledge integration in energy system research [32],
[33]. For this purpose, the authors developed a theoretical
model of hybrid data based on the fractal stratified model
(FS-model) of information space.\par
The hybrid data model is based on the development of
a system of ontologies of the remote sensing information
space, including a metaontology describing the layers of
the FS model and ontologies of individual layers (subject
areas).\par
As a result of ontological modeling, an ontological
space including a set of ontologies is created, which
should allow working not only with data, but also with
knowledge, including descriptions of scenarios of various
situations, models and software complexes, and integrate
them into the IT infrastructure of interdisciplinary research.\par
The Open Geospatial Consortium (OGC)
GeoSPARQL standard supports representing and
querying geospatial data on the Semantic Web [34],
[35]. GeoSPARQL defines a vocabulary for representing
geospatial data in RDF, and it defines an extension to the
SPARQL query language for processing geospatial data.
In addition, GeoSPARQL is designed to accommodate
systems based on qualitative spatial reasoning and
systems based on quantitative spatial computations.\par
Thanks to Semantic Web technology and ontology
engineering, as well as standardization processes for
ontology development in the web ontology language,
the problem of declarative knowledge representation has
been solved, which contributes to the understanding of
map objects and allows querying spatial data explicitly
represented in spatial data storage formats [36], [37].\par
However, subject domain formalization and ontology
engineering is only one step in intelligent systems design
technology and by itself is not sufficient for knowledgebased inference, since ontology engineering allows for
the description of declarative knowledge of subject domains, whereas procedural knowledge allows for the design of problem solvers and knowledge-based inference.\par
The above-mentioned possibilities of the technology
based on the semantic web have certainly contributed to
the development of geographic information systems with
the ability to process colossal volumes of crowdsourced
data. At the same time, decision making in problem
domains of human activity requires obtaining an intelligent reference, i. e. actually solving a problem when
the answer is not available in the datasets themselves
or represented knowledge in the current version of the
knowledge base or in the repository. A way of expressing
such a need is the question [38]–[40]. In the process
of communication there is always a context, which determines additional information that contributes to the
correct understanding of the meaning of the message.
Systems that are able to provide background information
on the user’s question belong to the class of intelligent
help systems.\par
In intelligent reference systems, the problem is formulated in the form of a question, and the answer to
the question requires specialized knowledge in science,
technology, art, craft or other fields of activity, which is
represented in knowledge bases. In other words, within
the framework of the considered technologies it is necessary to first generate knowledge of the problem domain
necessary for giving an answer. At the same time, the
capabilities of knowledge bases of intelligent systems
allow not only to represent and structure knowledge
about the surrounding world, but also to quickly obtain
and form this knowledge about it, thus satisfying the
information need of the user [41].\par
One of the key features of an intelligent system is that
the user has the ability to formulate his/her information
need. The peculiarity of information representation in
the knowledge bases of intelligent systems simplifies
the formation of the user’s information need, since the
presented information in the knowledge bases is already
structured and the relations defined on a certain concept,
in respect of which the question-problem situation is
solved, are known. In the work [42] it is shown that
the question-problem situation cannot be solved within
the framework of formal logic and the nature of the
question can be understood in the system of subjectobject relations. In connection with the fact that at
formation of knowledge bases of intellectual systems the
formation of subject-object relations within the given subject area takes place, thereby simplifying the expression
of information need by the user by means of knowledge
representation languages.\par
The proposed approaches to optimize information
retrieval currently lie in the development of questionanswer systems (QAS) in which user questions are
matched with the required information. Such systems
carry out a dialog between the user and the system in
the form of the procedure "QUESTION-Answer" in the
 \par
 }

}

\end{multicols}

\end{document}
\documentclass[10pt, a4paper]{article}

\usepackage{CJKutf8}
\usepackage[russian, english]{babel} %язык

\renewenvironment{itemize}{
    \begin{list}{\labelitemi}{
    \setlength{\topsep}{0pt}
    \setlength{\partopsep}{0pt}
    \setlength{\parskip}{0pt}
    \setlength{\itemsep}{0pt}
    \setlength{\parsep}{0pt}
    }
}{\end{list}}

\usepackage[utf8]{inputenc}
\usepackage{multicol} %колонки
\usepackage{setspace} %межстрочный интервал
\usepackage{ragged2e}% выравнивания текста по ширине в документе.
\usepackage{fancyhdr} %настройки верхнего и нижнего колонтитулов в документе.
\usepackage{titlesec} %стилей заголовков разделов в документе.
\usepackage{enumitem} %настройки списков в документе.
\usepackage{graphicx}%Вставка картинок правильная
\usepackage{float}%"Плавающие" картинки
\usepackage{wrapfig}%Обтекание фигур (таблиц, картинок и прочего)
\usepackage{hyperref}%Гиперссылки

\usepackage[left=1.9cm,right=1.9cm, top=2.2cm,bottom=2.5cm]{geometry}

\justifying % выравнивает текст по ширине.
\fancyhf{} %очищает все верхние и нижние колонтитулы.
\renewcommand{\headrulewidth}{0pt} % remove the header rule
\cfoot{\vskip -1.5cm \thepage} %устанавливает номер страницы в нижнем колонтитуле.

\linespread{0.8} %устанавливает межстрочный интервал в 0.84.


\setlength{\columnsep}{0.5cm}
\setcounter{page}{156}
\renewcommand{\thesection}{\Roman{section}} %устанавливает стиль нумерации разделов в виде заглавных римских цифр.

\titleformat{\section}{\footnotesize\centering\sc}{\thesection.}{0cm}{}[] %настраивает стиль заголовков разделов.

%\usepackage[unicode, pdftex]{hyperref} % Пакет для подключения ссылок





\begin{document}

\selectlanguage{english}
\fontsize{10}{14}\selectfont
\begin{multicols}{2}
\setlength{\parindent}{0.8cm}
\par
\setlength{\parindent}{0.3cm}
\fontsize{10}{15}\selectfont
In non-invasive diagnostics, sensor output signals have
different physical nature and correspondingly different
types of data representation - and the question arises
how to process them. This, in turn, necessitates the
use of ontological approach. It is proposed to use the
domestic technology of complex life cycle support of
semantically compatible intelligent computer systems of
new generation (Open Semantic Technology of Intelli-
gent Systems). Logical-semantic approach in diagnostic
tasks will allow to carry out differential diagnostics (to
put forward several diagnostic hypotheses).
\par The technological basis for the creation of an intelli-
gent diagnostic system is the technology of OSTIS.
\par The project of intellectual non-invasive diagnostics is
justified by the presence of technological basis of OSTIS,
application of logical-semantic approach in diagnostic
tasks, as well as different variants of non-invasive di-
agnostics.
\par In the future, one of the directions of development of
this system is the development of a personal medical as-
sistant for the patient, which is focused not only on early
detection of the disease, but also on recommendations to
the patient for possible additional examination.
\par The basis for the functioning of the personal medical
assistant is a decision support system. Obtaining knowl-
edge by the patient will contribute to the understanding
of his condition and its possible causes.
\par Important aspects of the system functioning are ac-
cumulation of data and knowledge, their systematization
and the possibility of system evolution. \newline

\begin{thebibliography}{21}
\fontsize{8}{5}\selectfont
\setlength{\parindent}{0.0cm}
\setlength{\parskip}{0.0cm}
     \bibitem[1]{e1}V. N. Rostovtsev, V. S. Ulashchik. New technology of physical
medicine. Zdravookhranenie, 2005, №5, pp. 10-14
     \bibitem[2]{e2}M. I.Silkou, R. M.Raviako, N. G.Lipnitskaya Multifunctional
device for non¬invasive measurement of functional condition of
patients. 6–th International Seminar on science and computing.
Moscow, 2003, pp. 488-493.
     \bibitem[3]{e3}V. L.Malykh. "Decision support systems in medicine". Software
Systems: Theory and Applications, 2019, vol. 10 no. 2(41), pp.
155-184.
     \bibitem[4]{e10}Efimenko I. V., Khoroshevsky V. F. Intelligent decision support
systems in medicine: retrospective review of the state of research
and development and prospects.Open semantic technologies for
designing intelligent systems, Iss. 1, pp. 251-260.
     \bibitem[5]{e4}Kobrinsky B. A. Artificial intelligence systems in medical prac-
tice: status and prospects. Bulletin of Roszdravnadzor. 2020. №3,
pp. 37-43.
     \bibitem[6]{e5}Sutton R. T. et al. An overview of clinical decision support
systems: benefits, risks, and strategies for success. NPJ digital
medicine, 2020, Vol.3, №1, P. 17
    \bibitem[7]{e6}Berseneva E. A., Mikhaylov D. Y. The experience of using intel-
ligent diagnostic decision support systems in a multidisciplinary
hospital, Ural Medical Journal, №05 (188) 2020, pp. 174-180.
    \bibitem[8]{e6}Shchekina E. N. Using a systematic approach to create decision
support systems in medicine (literature review). VESTRUCTURE
OF NEW MEDICAL TECHNOLOGIES, Vol. 11, №2 2017,
RUSSIA, TULA, pp. 356-364.
    \bibitem[9]{e6} — Smart system of personal health monitoring. Medelectron-
ics — 2020. Means of medical electronics and new medical
technologies : collection of scientific articles XII International
Scientific and Technical Conference, Minsk, December 10, 2020.
BSUIR, Minsk, 2020, pp. 198-203.
    \bibitem[10]{e6} Aminu E. F. et al. A review on ontology development methodolo-
gies for developing ontological knowledge representation systems
for various domains. International Journal of Information Engi-
neering and Electronic Business( IJIEEB), 2020, Vol. 12, №2,
pp. 28-39.
    \bibitem[11]{e6}Shunkevich, D. Ontological approach to the development of
hybrid problem solvers for intelligent computer systems. Open
Semantic Technologies for Intelligent Systems. Minsk, 2021,
Iss. 5, pp. 63-74.
    \bibitem[12]{e6}Technology of complex life cycle support of semantically compatible intelligent computer systems of new generation. Minsk,
Bestprint, 2023, 1064 P.
    \bibitem[13]{e6} — Medical diagnostics ontology for intelligent decision support
systems. Design Ontology, 2018, Vol. 8, №1(27), pp. 58-73.
    \bibitem[14]{e6} V. Rostovtsev Intelligent Health Monitoring Systems. Open Se-
mantic Technologies for Intelligent Systems. Minsk, 2023, Iss. 7,
pp. 237-239.
    \bibitem[15]{e6} — Bioimpedanceometry as a method of assessing the component
composition of the human body (literature review). Bulletin of
St. Petersburg State University. Medicine, 2017, Vol. 12, №4.
pp. 365-384.
    \bibitem[16]{e6}Lipnitskaya, N. G. Synthesis of the information-measuring de-
vices with application of the evolutionary calculations: Cand. Sci.
(Techn.) Dissertation: 05.13.05, Minsk, 2005, 115 P.
    \bibitem[17]{e6} — Method and device for detection of a diseased organ in a
patient by Zakharyin-Ged zones : patent. BY 10905. Published
30.08.2008.
    \bibitem[18]{e6}Boytsov I. V. Dynamic segmental diagnostics. Manual for doctors,
N.Novgorod, "Typography "Povolzhie", 2014, 460 P.
    \bibitem[19]{e6}V. N. Rostovtsev, A. K. Kalmanovich, A. O. Lukyanov Device
for recording and compensation of spectral-dynamic processes :
patent. BY 6321. Published 30.06.2010
    \bibitem[20]{e6}Boytsov I. V., Kononovich V. I. Meridian diagnostics: a modern
approach to research. Military Medicine, 2024, №1, pp. 60-70.
\end{thebibliography}


\setlength{\parindent}{0.8cm}
\par
\setlength{\parindent}{0.3cm}
\fontsize{10}{15}\selectfont
\selectlanguage{russian}
\begin{center}
\normalsize{\textbf{ОСНОВАНИЯ ИНТЕЛЛЕКТУАЛЬНОЙ
НЕИНВАЗИВНОЙ ДИАГНОСТИКИ}}
\fontsize{12}{15}\selectfont
\par Липницкая Н. Г., Ростовцев В. Н.
\end{center}
\fontsize{10}{13}\selectfont
\par В статье предложено обоснование направления ра
бот по созданию системы интеллектуальной неинва
зивной диагностики. В качестве ее основополагающих
составляющих определены два аспекта: технологи
ческий базис для разработки и различные варианты
неинвазивной диагностики.
\par Технологическим базисом для создания интеллектуальной диагностической системы является отечественная Открытая Семантическая Технология Интеллектуальных Систем (ОСТИС). Использование логико
семантического подхода в диагностических задачах позволит осуществлять дифференциальную диагностику (выдвигать несколько диагностических гипотез). Рассмотрено несколько направлений (вариантов) неинвазивной диагностики: функционально
спектральная диагностика (ФСД-диагностика), биоим-педансный анализ, предварительная диагностика на
основе оценки основных параметров функционального
состояния, диагностика по зонам Захарьина-Геда, диагностика по методу Накатани, частотно-резонансная
диагностика.

\begin{flushright}
    \par
    \par Received 13.03.2024
\end{flushright}

\clearpage

\end{multicols}
\setlength{\parindent}{0.3cm}
\fontsize{10}{15}\selectfont
\selectlanguage{russian}
\begin{center}
\fontsize{22}{30}\selectfont
\textbf{Integration and Standardization
\\
in New Generation Intelligent Medical Systems 
\\
Based on OSTIS Technology}
\end{center}
\setlength{\parindent}{0.8cm}
\par
\setlength{\parindent}{0.3cm}
\begin{center}
\fontsize{10}{13}\selectfont
Veronika Krischenovich, Daniil Salnikov, Vadim Zahariev
\\
\textit{Belarusian State University of Informatics and Radioelectronics}
\\
Minsk, Belarus
\\
\{krish, d.salnikov, zahariev\}@bsuir.by
\end{center}

\begin{multicols}{2}
\par \textbf{\textit{Abstract}—The article discusses the integration of international medical standards in Russia, Belarus, and Kazakhstan using semantic technologies. It describes OSTIS
(Open Semantic Technology for Intelligent Systems) — an
open project for creating common semantic technologies for
the component design of intelligent systems. An example
of the integration of various medical records standards in
intelligent medical systems is given. The advantages of such
integration are the improvement of the exchange of medical
information, simplification of the diagnosis and treatment
process, and the possibility of creating a unified medical
space within the region.}
\par \textbf{\textit{Keywords}—ontology modeling, OSTIS, medical information systems, standardization of medical data, integration
of medical systems, International Classification of Diseases
(ICD), flexibility and adaptability in healthcare systems.}

    
\section{Introduction}

\par The possibilities of application of new modern technologies in human life are growing day by day. Implementation of artificial intelligence in medical systems
plays an important role in medical data processing.
The huge amount of data processed in modern medical
systems can be quickly analyzed and processed only
using the artificial intelligence (AI) element that is being
implemented in modern computer technologies. Processing medical data using AI enables the identification of
diseases and the patterns and trends of their development,
helps to identify pathologies and risks of disease, as well
as forecasting the spread of infections and predicting epidemics. This helps doctors and healthcare organizations
in general to make informed decisions [1]–[3].
\par The implementation of AI in medical systems can
solve both complex and simpler problems. The complex challenges include the implementation of AI in
robotic surgery. The application of AI in robotic surgery
could significantly accelerate the development of surgical
robots to perform complex surgeries with a high level of
precision, which in turn will reduce risks and shorten
patient recovery time after surgery. More straightforward challenges include the personalization of medicine,
namely taking into account individual patient characteristics: genetic data and biomarker analysis, when developing personalized treatment plans and systematizing
patients’ medical histories [4], [5].
\par Electronic medical histories of patients, as well as dig-
ital data from medical examinations, patient monitoring
data from medical devices, are part of a unified medical
decision support system called a medical information
system (MIS). MIS is a key element of the medical
system. MIS providing automation of document flow
and accounting in medical institutions has an important
role in modern medicine. And the introduction of AI
in MIS allows to move these systems to a new level of
development.

\section{Problematics of modern medical information
systems}

\par MIS is an information system designed for processing,
accumulation, storage and retrieval of a patient’s electronic medical record. MIS can be classified depending
on the area of activity of medical institutions. For example, MIS for hospital usually collects and processes
information from all blocks of the information system,
including operating and resuscitation units. They may
include modules for managing patients, staff, equipment,
as well as for monitoring the performance of medical
procedures and treatment. And MIS for outpatient clinics,
in turn, usually focus on automating processes such as
making appointments, working with waiting lists, and
maintaining patient registries. They may also include
functions for managing doctors’ schedules, tracking patients’ medical histories, and sharing medical information
between different specialists and departments [6].
\par Thus, it can be concluded that MISs vary depending
on the specific needs of the healthcare facility, but all
systems will perform functions such as completing a
patient’s medical record and tracking medical history.
\par The first thing patients encounter when moving from
one health care facility to another is the need to re-
fill out their data for their medical records. Although
personal data remains unchanged, the medical history
may be incomplete because the patient cannot access all
the data about all the examinations they have undergone.
\clearpage

This can make diagnosis difficult and distort the overall
picture. It is worth noting that the more systems in which
a patient enters their personal data, the greater the risk
of this data being leaked.
\par Protection of patients’ personal data is an important
element of MIS operation. When implementing an information system, the staff of a health care institution should
make certain efforts, for example, they should follow
the algorithm of working with the selected information
system, enter information into the system using available
templates and forms and consistently maintain electronic
medical records.
\par In 2020, during a roundtable discussion at the BELTA
press center, representatives of the Ministry of Health
and practical medicine discussed the promising directions of Belarusian e-health. And even earlier, in March
2018, the Concept of e-health development of the Republic of Belarus for the period until 2022 was developed
and approved. The purpose of which was to develop e-
health and create a centralized health information system
(CHIS) for the formation of a unified information archive
of patients and exchange of medical data [7].
\par The activity of the CHIS aims to improve the avail-
ability and quality of health care by assisting in clinical
decision-making, improving the quality and efficiency of
management decisions based on statistical and analytical
data.
\par CHIS consists of functional and support subsystems
and other subsystems. Supporting subsystems include
software and hardware complexes, information protection
system and subsystems that ensure proper technical functioning and interdepartmental information interaction of
the CHIS. One of the subsystems included in the CHIS
is the electronic medical record of the patient and the
clinical decision support system.
\par Thus, we can conclude that the electronic medical
record is a MIS, a subsystem of the CHIS. And the main
obstacle to the creation of CHIS is the lack of standards
in the field of e-health and regulations for the exchange
of electronic medical information
\par As recommended by the World Health Organization
and the International Telecommunication Union, a national eHealth system requires the following components:
standards and interoperability (component); an enabling
environment (role); and standards that will ensure the
holistic and accurate capture and exchange of health
information across all health systems and services (func-
tional purpose).There are also a number of principles that
should be considered when developing a CHIS:

\begin{itemize}
    \item utilization of cloud computing technologies;
    \item use of open source software;
    \item service-oriented architecture, microservices, modularity, possibility to create additional services through open interfaces;
    \item elimination of duplication of engineering and telecommunication infrastructure;
    \item Web client technology;
    \item ensuring information security and information protection;
    \item scalability;
    \item simple and user-friendly interfaces, ergonomic and intuitive to use;
    \item single entry and repeated use of primary information;
    \item Interoperability of MIS with CHIS.
\end{itemize}
\par Thus, there is a need for technology that meets all
the requirements for the realization of CHIS and, in
particular, MIS.
\par The results of the implementation of the Concept in
eHealth is the creation of the following systems:
\begin{enumerate}
    \item National registers:
    \begin{itemize}
        \item State Register "Diabetes Mellitus";
        \item State register of persons exposed to radiation as a result of the Chernobyl catastrophe and other radiation accidents;
        \item Belarusian Kancer Register;
        \item Republican register of HIV-infected patients;
        \item Republican register "Tuberculosis".
    \end{itemize}
    \item Medical information systems:
    \begin{itemize}
        \item AIS "Electronic Prescription". This system is designed to automate the process of prescription writing and control over its fulfillment.
        \item RSTMC (Telemedicine). Telemedicine system allows doctors to counsel patients from a distance using video, audio or messenger chat.
        \item IAS "Zdravookhranenie". This system is designed to automate the recording and analysis of medical information, including data on the health status of patients.
        \item IAS "Drug Supply". This system is designed to automate the process of planning and control of centralized procurement of medicines for healthcare organizations.
    \end{itemize}
\end{enumerate}
\par All these systems are aimed at improving the quality and efficiency of medical care through the use of
modern information technologies. They also promote
standardization and centralization of medical information, which facilitates data exchange and collaboration
between different specialists and medical institutions. In
addition, these systems help to ensure the security and
confidentiality of medical data.
\par  There are a number of foreign analogs of MIS. Here
are a few MIS popular in Russia:
\begin{enumerate}
    \item ArchiMed+ is a versatile medical software that is suitable for private physicians, medical centers, dental offices, and chain clinics. ArchiMed+ is easily scalable, offers many integrations including third-party labs, labeling system, telemedicine and more.
\end{enumerate}

\end{multicols}
\clearpage
Новый список
\begin{enumerate}
    \item Что-то первое
    \item Что-то второе с подсписком
    \begin{itemize}
        \item бла-бла1
        \item бла-бла2
    \end{itemize}
    \item Что-то третье
\end{enumerate}

\clearpage

Страница с дополнениями всякими

\begin{figure}[H]
    \centering
    \includegraphics[width=15cm]{imagebanan.jpg}
    \caption{Пример картинки}
    \label{Fig. 1}
\end{figure}

\clearpage

Ссылка на банан\pageref{Fig. 1}

Ссылка на буквы \hyperref[Bukvi1]{ссылка}

Ссылка на буквы2 \hyperlink{bukvi2}{ссылка}

\clearpage

Буквы: ПАЩЗПЩШЗВПШАЫОМПУОАТУЦОПАЛОВРЫЛМОЛДВЛФЫОУМВОМОДФВЫЛДМВ\label{Bukvi1}

Буквы2: аЛПЩЗВОЫЗарзвыодыатвыдисшщйуйрзоащ \hypertarget{bukvi2}{\textbf{Якорь}}
\end{document}
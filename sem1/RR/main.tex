\documentclass[12pt]{article}
\usepackage{geometry}
\geometry{a4paper, margin=1in}
\usepackage{times}
\usepackage{setspace}
\usepackage[utf8]{inputenc}
\usepackage[russian]{babel}
\usepackage{titlesec}
\usepackage{fancyhdr}
\usepackage{hyperref}

% Настройка заголовков
\titleformat{\section}{\bfseries\fontsize{15}{17}\selectfont}{\thesection.}{1em}{}

\titleformat{\subsection}{\bfseries\fontsize{13}{14}\selectfont}{\thesubsection.}{1em}{}
\setstretch{1.15}
\renewcommand{\baselinestretch}{1.15}
\fancyhf{}
\rfoot{\thepage}

\begin{document}

\begin{center} 
   \Large Расчетная работа
   \end{center} 

\section{Тема}

Разработка программы решения теоретико-графовой задачи

\section{Цель}

Изучить основы теории графов, ключевые определения и понятия.

\section{Задача}

При выполнении расчетной работы необходимо разработать и реализовать программу на С/С++, которая решает выданную преподавателем теоретико-графовую задачу.

\section{Вариант}

Вариант 2(2):

Определить диаметр неориентированного взвешенного графа, заданного через матрицу смежности.

\section{Список ключевых понятий}

\begin{itemize}
\item Граф - это топологичекая модель, которая состоит из множества вершин и множества соединяющих их рёбер.
\item Неориентированный граф - граф, ни одному ребру которого не присвоено направление.
\item Матрица смежности графа — матрица, значения элементов которой характеризуются смежностью вершин графа. 
\item Взвешенный граф - граф, каждому ребру которого поставлено в соответствие некое значение (вес ребра).
\item Алгоритм Флойда-Уоршелла — это алгоритм для нахождения кратчайших путей между всеми парами вершин в графе.
\item Диаметр графа - максимум расстояния между вершинами для всех пар вершин.
\end{itemize}

\section{Тестовые примеры}

\subsection*{Пример 1:}

Входная матрица смежности:
\begin{verbatim}
0 1 0 1 1 0 0 0
1 0 1 0 1 1 1 0
0 1 0 1 0 0 0 1
1 0 1 0 1 0 1 0
1 1 0 1 0 1 0 1
1 1 0 0 1 0 1 0
0 1 0 1 0 1 0 0
0 0 1 0 1 0 0 0
\end{verbatim}
Матрица расстояний после алгоритма Флойда-Уоршелла:
\begin{verbatim}
0   4   5   6   2   3   6   4
4   0   4   3   6   5   2   5
5   4   0   7   3   8   6   1
6   3   7   0   6   6   1   8
2   6   3   6   0   5   7   2
3   5   8   6   5   0   5   7
6   2   6   1   7   5   0   7
4   5   1   8   2   7   7   0
\end{verbatim}

\begin{verbatim}
Диаметр графа: 8
\end{verbatim}

\subsection*{Пример 2:}

Входная матрица смежности:
\begin{verbatim}
0 1 0 1 0 0 0
1 0 1 0 1 0 0
0 1 0 0 0 1 0
1 0 0 0 0 0 1
0 1 0 0 0 1 1
0 0 1 0 1 0 1
0 0 0 1 1 1 0
\end{verbatim}
Матрица расстояний после алгоритма Флойда-Уоршелла:
\begin{verbatim}
0   1   3   4   4   6   8
1   0   2   5   3   5   7
3   2   0   7   5   5   9
4   5   7   0   8  10   6
4   3   5   8   0   2   4
6   5   5  10   2   0   6
8   7   9   6   4   6   0
\end{verbatim}

\begin{verbatim}
Диаметр графа: 10
\end{verbatim}

\subsection*{Пример 3:}

Входная матрица смежности:
\begin{verbatim}
0 1 0 1 0 0 1 0
1 0 1 0 0 0 0 1
0 1 0 0 1 0 0 0
1 0 0 0 0 1 0 1
0 0 1 0 0 0 1 0
0 0 0 1 0 0 0 1
1 0 0 0 1 0 0 0
0 1 0 1 0 1 0 0
\end{verbatim}
Матрица расстояний после алгоритма Флойда-Уоршелла:
\begin{verbatim}
0   8  10   9   9  14   5  11
8   0   2  11   8  12  12   9
10   2   0  13   6  14  10  11
9  11  13   0  18   5  14   2
9   8   6  18   0  20   4  17
14  12  14   5  20   0  19   3
5  12  10  14   4  19   0  16
11   9  11   2  17   3  16   0
\end{verbatim}

\begin{verbatim}
Диаметр графа: 20
\end{verbatim}

\subsection*{Пример 4:}

Входная матрица смежности:
\begin{verbatim}
0 1 0 1 1 0 1 0
1 0 1 0 1 0 0 1
0 1 0 0 1 0 1 0
1 0 0 0 0 1 0 1
1 1 1 0 0 0 1 0
0 0 0 1 0 0 0 1
1 0 1 0 1 0 0 1
0 1 0 1 0 1 1 0
\end{verbatim}
Матрица расстояний после алгоритма Флойда-Уоршелла:
\begin{verbatim}
0   4   8   1   4   3   7   7
4   0   6   5   7   7   7   7
8   6   0   9   4  11   1   5
1   5   9   0   5   2   8   6
4   7   4   5   0   7   3   7
3   7  11   2   7   0  10   8
7   7   1   8   3  10   0   4
7   7   5   6   7   8   4   0
\end{verbatim}

\begin{verbatim}
Диаметр графа: 11
\end{verbatim}

\subsection*{Пример 5:}

Входная матрица смежности:
\begin{verbatim}
0 1 1 0 0 0
1 0 0 1 0 1
1 0 0 0 1 0
0 1 0 0 1 1
0 0 1 1 0 0
0 1 0 1 0 0
\end{verbatim}
Матрица расстояний после алгоритма Флойда-Уоршелла:
\begin{verbatim}
0   7   9  13  10  14
7   0  11   6  10   7
9  11   0   5   1  10
13   6   5   0   4   5
10  10   1   4   0   9
14   7  10   5   9   0
\end{verbatim}

\begin{verbatim}
Диаметр графа: 14
\end{verbatim}

\section{Детализация преобразования входной конструкции в выходную}

Рассмотрим процесс преобразования входных данных на конкретном примере.
\setlength{\parskip}{0.2 cm} \par
\subsection{Входные данные:}

Файл adjacency.txt (матрица смежности):
\begin{verbatim}
0 1 0 1 
1 0 1 1 
0 1 0 1 
1 1 1 0
\end{verbatim}

Файл weights.txt (веса рёбер): 
\begin{verbatim}
3 7 1 2 2 7 1 1
\end{verbatim}

\subsection{Процесс преобразования}

\quad 1) Считывание матрицы смежности

2) Инициализация весов рёбер (начальное значение):

Изначально все веса устанавливаются в I, кроме диагональных элементов, которые равны 0.

3) Считывание и установка весов рёбер:

Считываются веса рёбер из weights.txt и устанавливаются на пересечении смежных вершин для обоих направлений рёбер. Обратные рёбра также учитываются (так как граф неориентированный). 

Начальная матрица весов рёбер:
\begin{verbatim}
0  3  I  7
3  0  1  2
I  1  0  2
7  2  2  0
\end{verbatim}

4) Алгоритм Флойда-Уоршелла:

Алгоритм использует три вложенных цикла для итерации по всем вершинам графа и обновления матрицы кратчайших путей. 

Внешний цикл по вершине k (вершина-посредник):
\begin{verbatim}
for (int k = 0; k < n; ++k)
\end{verbatim}

Средний цикл по вершине i (начальная вершина):
\begin{verbatim}
for (int i = 0; i < n; ++i)
\end{verbatim}

Внутренний цикл по вершине j (конечная вершина):
\begin{verbatim}
for (int j = 0; j < n; ++j)
\end{verbatim}

Для каждой пары вершин (i, j) проверяется, можно ли улучшить кратчайшее расстояние через вершину k:
\begin{verbatim}
if (dist[i][k] < INF && dist[k][j] < INF) {
    dist[i][j] = min(dist[i][j], dist[i][k] + dist[k][j]);
}
\end{verbatim}

Шаг 1. Проход через вершину 1 (k=0)

При проходе через вершину 1 нет более коротких путей, поэтому в матрице рёбер ничего не изменяется.

Шаг 2. Проход через вершину 2 (k=1)

Обновляются следующие пути: 

1 -> 3: 1 -> 2 -> 3 = 3 + 1 = 4

1 -> 4: 1 -> 2 -> 4 = 3 + 2 = 5

3 -> 1: 3 -> 2 -> 1 = 1 + 3 = 4

4 -> 1: 4 -> 2 -> 1 = 2 + 3 = 5

Обновленная матрица весов рёбер:
\begin{verbatim}
0  3  4  5
3  0  1  2
4  1  0  2
5  2  2  0
\end{verbatim}

Шаг 3. Проход через вершину 3 (k=2)

При проходе через вершину 3 нет более коротких путей, поэтому в матрице рёбер ничего не изменяется.

Шаг 4. Проход через вершину 4 (k=3)

При проходе через вершину 4 нет более коротких путей, поэтому в матрице рёбер ничего не изменяется.

5) Окончательная матрица кратчайших путей:
\begin{verbatim}
0  3  4  5
3  0  1  2
4  1  0  2
5  2  2  0
\end{verbatim}

Окончательная матрица кратчайших путей показывает минимальные расстояния между всеми парами вершин после применения алгоритма Флойда-Уоршелла.

6) Нахождение диаметра графа:

Диаметр графа — максимальное расстояние в матрице кратчайших путей.

Диаметр графа: 5

\subsection{Выходные данные:}

\begin{verbatim}
Диаметр графа: 5
\end{verbatim}

\setlength{\parskip}{0 cm} \par

\section{Вывод}

В ходе расчетной работы я изучила основы теории графов, ключевые определения и понятия; разработала алгоритм нахождения диаметра неориентированного взвешенного графа и реализовала его на языке C++. 

\section{Список использованных источников}

\begin{itemize}
\item Гладков, Л. А. Дискретная математика : учебное пособие / Л. А. Гладков, В. В. Курейчик, В. М. Курейчик ; под редакцией В. М. Курейчика. — Москва : ФИЗМАТЛИТ, 2014.
\item \href{https://ru.wikipedia.org/wiki/%D0%93%D0%BB%D0%BE%D1%81%D1%81%D0%B0%D1%80%D0%B8%D0%B9_%D1%82%D0%B5%D0%BE%D1%80%D0%B8%D0%B8_%D0%B3%D1%80%D0%B0%D1%84%D0%BE%D0%B2}{Свободная энциклопедия "Википедия". Глоссарий теории графов}
\item \href{https://habr.com/ru/companies/otus/articles/568026/}{Теория графов. Термины и определения в картинках}
\end{itemize}

\end{document}

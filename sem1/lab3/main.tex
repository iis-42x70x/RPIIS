\documentclass[10pt, a4paper]{article}
\usepackage{hyperref}
 \usepackage[unicode, pdftex]{hyperref}
\hypersetup{
    colorlinks=true,
    linkcolor=blue,
    filecolor=magenta,      
    urlcolor=cyan,
    pdftitle={Overleaf Example},
    pdfpagemode=FullScreen,
    }
\usepackage[russian, english]{babel} % установка языков
\usepackage{multicol} % деление на колонки
\usepackage{fancyhdr} % настройка колонтитулов
\usepackage{parskip} % настройка отступов абзаца
\usepackage{biblatex} % список литературы
\usepackage{mathtext} % Times New Roman
\usepackage{indentfirst} % отступ после заголовка секции
\usepackage{titlesec} % настройка заголовков
\usepackage[margin=0.1cm]{caption} % настройка описаний
\captionsetup[figure]{font=scriptsize} % шрифт описания фигуры
\usepackage[left=2cm,right=2cm, top=2cm,bottom=2cm]{geometry} % макет страницы
\fancyhf{} % очистка колонтитулов
\cfoot{\textbf{\thepage}} % жирные номера страниц
\pagestyle{fancy}
\renewcommand{\headrulewidth}{0pt} % удаление линии header
\linespread{0.99} % межстрочный интервал
\setlength{\columnsep}{0.4cm} % расстояние между столбцами
\setlength{\parskip}{0pt} % вертикальный отступ абзаца
\setlength{\parindent}{0.5cm} % горизонтальный отступ абзаца
\setlength{\textfloatsep}{\theintvl\curtextsize} % отступ после картинок
\setcounter{page}{126} % нумерация страниц
\setcounter{figure}{7} % начальное значение нумерации фигур
\titleformat{\section}{\normalsize\centering}{\thesection. }{0cm}{}[] % стиль заголовков разделов
\titlespacing*{\section} % отступ возле секций
{0pt}{0.4cm}{0.3cm}
\renewcommand{\thesection}{\Roman{section}} % римские цифры
\setcounter{section}{6} % нумерация секций

\begin{document}
\begin{multicols}{2}
\noindent education should not just provide knowledge in individual
sciences, but educate a cognitive personality interested
in knowledge, its development, creativity, and interaction
with the environment, be it intellectual systems or people.
Interoperable intelligent training systems based on new
generation computer systems can play a significant role
in this.

The article presents an analysis of the importance and
ways of developing cognitive abilities, emotional intelligence and the properties of interoperability at school
age. Some proposals are presented for the introduction of
these methods and methods into the educational process
of secondary school. Some descriptions of basic concepts, processes and methods of the subject area “human
interoperability” is proposed for their formalization and.

\begin{center} 
\paragraph{}
Acknowledgment
\end{center}
\paragraph{}
The authors would like to thank the scientific teams of
the departments of intelligent information technologies of
the Belarusian State University of Informatics and Radioelectronics and the department of intelligent systems of
the Belarusian State University for their assistance in the
work and valuable comments.
\begin{footnotesize}
\begin{thebibliography}{20}
    \bibitem{book1} A. N. Leontiev Psikhologicheskie osnovy razvitiya rebenka I
obucheniya (digest) // Izdatelstvo “Smysl”, 2009, P. 426 (in
Russian)
    \bibitem{book2} Available at: linkedin.com/in/tatsiana-kandrashova-businessanalyst-a906b9131. Published February 24, 2021 (access date
2024-03-03).
    \bibitem{book3}  Zhuravlev M. S. Interoperabelnost kak factor razvitiya prava
v sfere elektronnogo zdravookhraneniya. Pravo. Zhurnal
Vysshey Shkoly ekonomiki. 2019. №3. (in Russian) Available
at: https://cyberleninka.ru/article/n/interoperabelnost-kak-faktorrazvitiya-prava-v-sfere-elektronnogo-zdravoohraneniya (access
date: 2024-03-03).
    \bibitem{book4} American Psychological Association Dictionary of Phychology.
American Psychological Association (electronic resource). Available at: https://dictionary.apa.org/cognitive-ability (access date
2024-03-20)
    \bibitem{book5} Steven Ramage, Jenna Slotin Why people are essential in
data interoperability. Global Partnership for Sustainable Development Data (electronic resource). Published August 25,
2021. Available at: https://www.data4sdgs.org/blog/why-peopleare-essential-data-interoperability (access date 2024-03-20)
    \bibitem{book6} World Development Report 2021. The World Bank (IBRD-IDA)
(electronic resource). Available at: https://wdr2021.worldbank.
org/stories/the-social-contract-for-data/ (access date 2024-03-20)
    \bibitem{book7}  What is emotional intelligence and how does it apply to the workplace? Mental Health America (electronic resource). Available
at: https://mhanational.org/what-emotional-intelligence-and-howdoes-it-apply-workplace (access date 2024-03-24)
    \bibitem{book8}  Wolf, Julia. (2022). Implications of pretend play for Theory of
Mind research. Synthese. 200. 10.1007/s11229-022-03984-5.
    \bibitem{book9} Aydin, Utkun & Ozgeldi, Meric. (2019). Unpacking the Roles
of Metacognition and Theory of Mind in Turkish Undergraduate
Students’ Academic Achievement: A Test Of Two Mediation
Models. 21. 1333-1365. 10.15516/cje.v21i4.3303.
    \bibitem{book10}  Smogorzewska, Joanna & Grzegorz, Szumski & Bosacki, Sandra & Grygiel, Pawel & Karwowski, Maciej. (2022). School
Engagement, Sensitivity to Criticism and Academic Achievement
in Children: The Predictive Role of Theory of Mind Short
title: Cognitive consequences of ToM development. Learning and
Individual Differences. 93. 10.1016/j.lindif.2021.102111.
    \bibitem{book11}  Alloway TP, Alloway RG. Investigating the predictive roles of
working memory and IQ in academic attainment. J Exp Child
Psychol. 2010 May;106(1):20-9. doi: 10.1016/j.jecp.2009.11.003.
Epub 2009 Dec 16. PMID: 20018296.
    \bibitem{book12} Luna B., Tervo-Clemmens B., Calabro F. J. Considerations when
characterizing adolescent neurocognitive development. Biological
psychiatry, 2021, Vol. 89, №2, pp. 96-98.
    \bibitem{book13}  Kail R. V., Ferrer E. Processing speed in childhood and adolescence: Longitudinal models for examining developmental change.
Child development, 2007, Vol. 7, №6, pp. 1760-1770.
\bibitem{book14} Susan M Sawyer et al. The age of adolescence. Prof, Susan M
Sawyer MD, Peter S Azzopardi, PhD, Dakshitha Wickremarathne,
MDS, Prof George C Patton, MD. The Lancet Child & Adolescent
Health. Vol. 2, Iss. 3, pp. 223-228, March 2018. DOI: https://doi.
org/10.1016/S2352-4642(18)30022-1
\bibitem{book15} Tervo-Clemmens, B., Calabro, F. J., Parr, A. C. et al. A canonical
trajectory of executive function maturation from adolescence to
adulthood. Nat Commun 14, 6922 (2023). https://doi.org/10.1038/
s41467-023-42540-8
\bibitem{book16} GOST R 55062-2021 Natsionalny standart Rossiyskoy Federatsii
“Informatsionnye tekhnologii. Interoperabelnost” OCS 35.240.50.
Date of introduction 2022-04-30
\bibitem{book17} Sravnitelnye harakteristiki neyrosetey i ih primenenie v obrazovanii [Comparative characteristics of neural networks and their application in education]. Available at: https://habr.com/ru/articles/
778044/. Published 2023-12-04. (access date 2024-03-04)
\bibitem{book18} Erik Ofgang «Gemini: Teaching With Google’s Latest AI».
https://www.techlearning.com/news/gemini-teaching-withgoogles-latest-ai. Published 20 December 2023 (access date
04-03-2024)
\bibitem{book19} TTehnologija kompleksnoj podderzhki zhiznennogo cikla semanticheski sovmestimyh intellektual’nyh komp’juternyh sistem
novogo pokolenija [Technology of complex life cycle support
of semantically compatible intelligent computer systems of new
generation]. Minsk, Bestprint, 2023, P. 1064 (in Russian)
\bibitem{book20} Kozlova E. I. Grakova N. V., Golovaty A. I. Automation of educational activities within the OSTIS Ecosystem. Open semantic
technologies for intelligent systems: scientific papers collection,
Minsk, BSUIR, 2023, Iss. 7, pp. 257-262.
\end{thebibliography}
\end{footnotesize}
\begin{otherlanguage}{russian}
\begin{center}
\vspace{0.3cm}
\small{\textbf{ИНТЕРОПЕРАБЕЛЬНОСТЬ КАК
ВАЖНЕЙШИЙ КОМПОНЕНТ
ИНТЕЛЛЕКТУАЛЬНОЙ ОБРАЗОВАТЕЛЬНОЙ
СРЕДЫ В СРЕДНЕЙ ШКОЛЕ}}
\vspace{0.1cm} \par
\large{Козлова Е. И., Головатый А. И.}
\end{center}

В работе представлены некоторые результаты анализа роли развития интероперабельности, когнитивных
способностей и эмоционального интеллекта у детей в
современной школе. Обсуждается важность и способы
внедрения технологических средств с возможностями
взаимодействия и обмена данными для оптимизации
образовательного процесса. Также рассматривается
значимость развития когнитивных способностей и эмоционального интеллекта учащихся и влияние этого на
их академические достижения и социальную адаптацию.
\end{otherlanguage}
\begin{flushright}
Received 24.03.2024
\end{flushright}
\end{multicols}
\begin{center}
\Huge{\textbf{\textit{OSTIS Glossary} — the Tool to Ensure
Consistent and Compatible Activity for the
Development of the New Generation Intelligent
Systems}}
\end{center}
\vspace{0.4cm}
\begin{center}
\center{Nikita Zotov, Tikhon Khodosov, Mikhail Ostrov, Anna Poznyak, Ivan Romanchuk,
Kate Rublevskaya, Bogdan Semchenko, Daria Sergievich, Artsiom Titov, Filip Sharou}
\center{\textit{Belarusian State University of}}
\center{\textit{Informatics and Radioelectronics}}
\center{Minsk, Belarus}
\center{Email:\href{ n.zotov@bsuir.by}}
\center{}
\end{center}
\begin{multicols}{2}

    \textbf{\textit{Abstract}—This paper includes a detailed analysis of the
problems of organising various types of collective activities,
a comparative analysis of current solutions to ensure the
consistency and compatibility of information from different
knowledge areas, as well as an analysis of methods and technologies for creating unified information spaces to ensure
consistent and compatible storage, processing, accumulation
and dissemination of knowledge. The paper proposes one
of the options for realising a unified information resource
to ensure consistent and compatible activities in the development of new generation intelligent computer systems
— the \textit{OSTIS Glossary}. It describes its structure, rules of
structuring, placement and identification of knowledge in
it, as well as its operating principles.} \par
\textbf{\textit{Keywords}—problem of mutual understanding, knowledge unification, knowledge convergence and integration,
knowledge consistency, semantic knowledge compatibility,
knowledge standardisation, interdisciplinary synthesis, consistency and compatibility of activities, Artificial Intelligence, semantic knowledge representation, semantic web,
knowledge base, intelligent system, scientific knowledge
portals,\textit{ OSTIS Glossary, OSTIS Standard}} \par
\setcounter{section}{0}
\section{Introduction}

In the era of information society, the problem of 
\underline{mutual} understanding between people is becoming more
and more acute. Due to the existing inconsistency in
the definition of terms of concepts, people not only
do not understand each other, but (!) also do not have
the necessary means to communicate with computer
systems and create collectives of computer systems that
understand each other [1], [2].

First of all, the reason for this problem lies in the form
of information representation and the means by which
this information is presented, accumulated, processed,
distributed and visualised [1], [3], [4]. As the amount of
information increases, not only the number of different
forms of representing this information [5] grows, but also
the number of different tools and methods to support
them.

In addition, due to the rapid development of information technology and the emergence of new fields, knowledge can quickly become outdated or may not correspond
to reality and current needs. Therefore, it is important
to constantly update and clarify terminology, establish
common standards and rules for the use of terms to
reduce the likelihood of inconsistency of concepts.

At the current stage of information technology development, the problem of information inconsistency is solved
with the help of integrated repositories in the form of
reference books, encyclopaedias, standards and online
resources. However, even with the use of these tools, the
problem of conceptual inconsistency remains relevant for
several [1] reasons:
\begin{itemize}
    \item In different fields of knowledge, terms of concepts
often have different meanings, leading to misunderstandings between the people using these terms [5].
    \item The understanding of the terms of concepts may
vary from person to person depending on their
experience, education and culture.
    \item With the development of the internet and digital
technology, information has become more accessible. Among the variety of information available, it
is often difficult to understand which term is used
in which context, what its exact meaning is, and to
which field it belongs.
    \item There are many sources of information, some of
which may be inaccurate, distorted or misinforming
[6].
    \item Modern natural languages are constantly evolving,
the terms of concepts in them are rapidly changing
their meanings or acquiring new ones depending on
context and usage, which complicates the task of
creating information resources, such as encyclopaedias or reference books, which could fully reflect
the unified meaning of terms [1].
    \item There is a problem of translating terms of concepts
from one language to another, which also leads to
misunderstanding and misuse of these terms.
\item Concepts are described and represented in different
ways, which makes it difficult for both humans and
computer systems to use them in problem solving
[7], [3].
\end{itemize}
Most often the main problem is not the concepts
themselves, but (!) the terms of the concepts with which
we name these concepts. All the above problems are
related not so much to the current capabilities of the
technology as to the current state of modern information
technologies and the means realised by them. To solve
these problems, it is necessary to switch to methods and
technologies of a new level [3], namely:
\begin{itemize}
    \item  to develop standards for the \underline {unified} unified interpretation
of concepts and their terms in different fields of
knowledge;
    \item to create and implement \underline {accessible}, \underline {unified 
integrated},  information resources that are continuously updated and adaptable to changing meanings
and contexts of the terms of concepts;
    \item to develop the conditions and capabilities to build
collectives of people and computer systems and to
enable them to enhance their  \underline {understanding}, agreement and coordination capabilities; 
    \item  to create and implement human-centred systems that
reduce the requirements and improve the conditions
for their adaptation adaptation and knowledge acquisition.
\end{itemize}
\par
In simple words, to solve these problems it is necessary
to comprehensively standardise the existing terminologies [3], [4] used in various fields of knowledge and to
create a unified information space for quick access to
information in it, as well as the possibility of using and
processing this information not only by a human but also
by computer systems [4], [8].

The objective of this paper is to address these issues
by creating a single \textbf{\textit{integrated glossary}} that can be used
to:
\begin{itemize}
    \item to provide a \underline {single} source for the interpretation of
concepts from different fields of knowledge;
    \item to provide a \underline {single} source for obtaining relevant and
reliable information;
    \item to \underline {integrate} knowledge from different information
sources;
    \item to \underline {systematise} concepts and the links between them
in the form of hierarchies of these concepts;
    \item to \underline {unify} the representation form of concepts and
their terms, i.e. to describe concepts in the same
language understandable \underline {both} to a human and a
computer system;
    \item to \underline {standardise} the descriptions of these concepts and
their terms by introduction and consistent use of
common rules for their identification, specification
and placement;
    \item to provide collective \underline {consistency} and \underline {supplement} of
existing concepts and introduction of new ones;
    \item to provide\underline { open access}  to all knowledge and facilitate its exchange;
    \item and, finally, to develop semantically compatible intelligent systems of various kinds. 
\end{itemize}

Such an integrated glossary needs to be developed as
an intelligent system, which will provide:
\begin{itemize}
    \item usability of this glossary, for example, will allow
to explain the difference between terms, to give
advice, to automate (!) the search for synonyms and
homonyms on the basis of some secondary features,
and so on;
\item convenience of this glossary development, which
will allow not only to collectively develop such a
glossary, but (!) also will help the system to improve
itself.
\end{itemize}

In the next section we will consider in detail the main
problems associated with the organisation of various
types of collective activity and the problems associated
with the representation, processing, accumulation and
transmission of information, since these are the problems
that are the object of study in this paper.
\section{Analysis of modern solutions to the problems of
consistency and compatibility of different types of
activities}

\textit{A. History of the development of the problem of consistency and compatibility of different activities}

To understand modern problems of consistency and
compatibility of different types of activities it is necessary to consider the problems associated with the presentation, processing and application of different kinds
of information, namely, it is necessary to study the
history of the development of society in the direction
of improving the ways of interaction between people,
organisation of their activities and ways of transferring
and accumulating knowledge from the older generation
of people to the new generation.

As far back as in ancient times, people began to face
the problem of understanding each other. First of all,
the need to understand each other was the need for
self-preservation (survival) in the environment of similar
people. The only way of communication between people
for a long time remained communication by means of
gestures, pictures and other similar means. For a long
time people exchanged knowledge with the help of these
means. And not all gestures came only from the man
himself. Man also perceived the gestures of nature and
made strategically important decisions based on them.
With the development of mental capabilities and the
increasing needs of each individual, people began to look
for more flexible forms to realise their self-preservation
(survival) in nature. People began to realise that it is
possible to negotiate with each other to achieve a certain
\end{multicols}


\end{document}

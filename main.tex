\documentclass{article}
\usepackage{enumitem}
\usepackage[russian]{babel}
\usepackage{ragged2e}
\usepackage{multicol}
\setlength { \columnsep }{ 2cm }
\usepackage[left=5cm,right=1.9cm, top=2.2cm,bottom=2.5cm]{geometry}
\linespread{0.84} 

\setcounter{page}{60}

\begin{document}
\begin{multicols}{2}
\begin{itemize}
\sloppy
\item[$\Rightarrow$] \quad
\textit{commmand to call an agent*: 
[components search -class concept\_reusable\_kb\_component]} 
     \leftskip=-1em
\item[$\Rightarrow$] \quad
\textit{result*:} \par 
      \leftskip=0pt
[Found all Knowledge Base components 
 whose specifications have been installed] \par
\leftskip=-3em
 \item\quad\textit{ Installing the Knowledge Base component} \par
  \leftskip=0pt
  \leftskip=-1em
 \item[$\Rightarrow$] \quad
 \textit{sc-agent*:} \par
   \leftskip=0pt
 [ScComponentManagerInstallAgent] \par
 \leftskip=-1em
 \item[$\Rightarrow$] \quad
 \textit{commmand to call an agent*:
[components install --idtf
part\_polygons]} \par
 \item[$\Rightarrow$] \quad
 \textit{result*} \par
 \leftskip=0pt
 [A Knowledge Base component in
the form of subject domain of
polygons was established.] \par
\leftskip=-1em
\item[$\Rightarrow$] \quad
\textit{note*:} \par
\leftskip=0pt
[After performing this step, we
can find the concept "multiple"
in the web interface and browse
its semantic neighbourhood. But
it is worth noting that the subject
domain of triangles, which is a
private subject domain of polygons, is empty.] \par
\leftskip=-3em
\item\quad\textit{Installing the Knowledge Base component} \par
\leftskip=-1em
\item[$\Rightarrow$] \quad
\textit{sc-agent*:} \par
\leftskip=0pt
[ScComponentManagerInstallAgent] \par
\leftskip=-1em
\item[$\Rightarrow$] \quad
\textit{commmand to call an agent*:
[components install --idtf
part\_triangles]} 
\item[$\Rightarrow$] \quad
\textit{result*:} \par
\leftskip=0pt
[The Knowledge Base component
is installed in the form ofof subject domain of triangle] \par
\leftskip=-1em
\item[$\Rightarrow$] \quad
\textit{note*:} \par
\leftskip=0pt
[After performing this step, we
can find the concept "triangle"
in the web interface and browse
its semantic neighbourhood. It is
worth noting that the subject domain of triangles, which is a private subject domain of polygons,
is fully described and compatible
with the subject domain of polygons.] \par
\leftskip=-3em
\item\quad\textit{Creating two sets of triangles} \par
\leftskip=-1em
\item[$\Rightarrow$] \quad
\textit{note*:} \par
\leftskip=0pt
[At this step it is necessary to find
the class "triangle" in the webinterface, create two sets of triangles and add elements to them.

\columnbreak

It is necessary to specify that the
sets and their elements belong to
the class "triangle".] 
\leftskip=-1em
\item[$\Rightarrow$] \quad
\textit{example*:} \par
\leftskip=0pt
\textit{[triangles\_1 = \{ABC, CDE, XYZ\},
triangles\_2 = \{MNK, CDE,
XYZ\}]} 
\leftskip=-1em
\item[$\Rightarrow$] \quad
\textit{note*:} \par
\leftskip=0pt
[After performing this step, you
can check that no operations on
sets can be performed now. This
can be verified by right-clicking
on the node "triangles\_1".] \par
\leftskip=-3em
 \item\quad\textit{Search for all available problem solver
components in the library} \par
\leftskip=-1em
\item[$\Rightarrow$] \quad
\textit{sc-agent*:} \par
\leftskip=0pt
[ScComponentManagerSearchAgent] \par
\leftskip=-1em
\item[$\Rightarrow$] \quad
\textit{commmand to call an agent*:
[components search  --class concept\_reusable\-ps\_component]} 
\item[$\Rightarrow$] \quad
\textit{result*:} \par
\leftskip=0pt
[Found all components of the problem solver whose specifications
are installed.] \par
\leftskip=-3em
  \item\quad\textit{Installing the components of the problem
  solver} \par
  \leftskip=-1em
\item[$\Rightarrow$] \quad
\textit{sc-agent*:} \par
\leftskip=0pt
[ScComponentManagerInstallAgent] \par
\leftskip=-1em
\item[$\Rightarrow$] \quad
\textit{commmand to call an agent*: 
[components install --idtf
agent\_of\_finding\_intersection\_of\_sets]}
\par
\item[$\Rightarrow$] \quad
\textit{result*:} \par
\leftskip=0pt
[A problem solver component for
finding the intersection of two
sets is established.] \par
\leftskip=-1em
\item[$\Rightarrow$] \quad
\textit{note*:} \par
\leftskip=0pt
[After this step, you can check
that you can now perform an
operation on sets. In the web
interface, search for the concept
"installed components" and select
the node of the desired agent
\textit{agent\_of\_finding\_intersection\_of\_sets)}
and run the set intersection agent
using the example of two
previously created triangle
sets. The intersection of the
two sets will be found. But it
should be noted that this way of
launching the agent is long and
inconvenient.] \par
\leftskip=-3em
  \item\quad\textit{Search for all available interface
components in the library} \par
\leftskip=0pt


\newpage

\leftskip=-1em
\item[$\Rightarrow$] \quad
\textit{sc-agent*:} \par
\leftskip=0pt
[ScComponentManagerSearchAgent] \par
\leftskip=-1em
\item[$\Rightarrow$] \quad
\textit{commmand to call an agent*:
[components search class concept\_reusable\_interface\_component]}
\item[$\Rightarrow$] \quad
\textit{result*:} \par
\leftskip=0pt
[Found all interface components
whose specifications have been
downloaded.] \par
\leftskip=-3em
\item\textit{Installing the user interface component} \par
\leftskip=-1em
\item[$\Rightarrow$] \quad
\textit{sc-agent*:} \par
\leftskip=0pt
[ScComponentManagerInstallAgent] \par
\leftskip=-1em
\item[$\Rightarrow$] \quad
\textit{commmand to call an agent*:
[components install −−idtf 
menu\_of\_agent\_of\_finding\_intersection\_of\_sets]}
\item[$\Rightarrow$] \quad
\textit{result*:} \par
\leftskip=0pt
[Installed an interface component
for an agent to find the intersection of two sets.] \par
\leftskip=-1em
\item[$\Rightarrow$] \quad
\textit{note*:} \par
\leftskip=0pt
[After this step, the intersection
finder can be invoked using a
button in the interface, which is
much faster and more convenient
than the first method. This can be
checked by calling the agent to
find the intersection of two sets
using the example of triangle sets
(triangles\_1 and triangles\_2).] \par
\leftskip=-3em
\item\textit{Setting a logical formula component} \par
\leftskip=-1em
\item[$\Rightarrow$] \quad
\textit{sc-agent*:} \par
\leftskip=0pt
[ScComponentManagerInstallAgent] \par
\leftskip=-1em
\item[$\Rightarrow$] \quad
\textit{commmand to call an agent*:
[components install −−idtf
lr\_about\_isosceles\_triangle]} \par
\item[$\Rightarrow$] \quad
\textit{result*:} \par
\leftskip=0pt
[Established a component with a
logical formula for determining
whether a triangle is isosceles or
not.] \par
\leftskip=-1em
\item[$\Rightarrow$] \quad
\textit{note*:} \par
\leftskip=0pt
[If you go to the web interface
after performing this step, create
the necessary fragment for the
geometry logic formula parcels
and try to run the logic output,
it fails because the logic output
component is missing.] \par
\leftskip=-3em
\item\textit{Setting the logic inference component}
\item[$\Rightarrow$] \quad
\leftskip=-1em
\textit{sc-agent*:} \par
\leftskip=0pt
[ScComponentManagerInstallAgent]

\columnbreak

\leftskip=-1em
\item[$\Rightarrow$] \quad
\textit{commmand to call an agent*:
[components install −−idtf
scl\_machine]}
\item[$\Rightarrow$] \quad
\textit{result*:} \par
\leftskip=0pt
[Logic inference machine is installed.] \par
\leftskip=-1em
\item[$\Rightarrow$] \quad
\textit{note*:} \par
\leftskip=0pt
[After performing this step go to
the web-interface, create the necessary fragment to send a logical
formula on geometry and try to
run the logical output, then the
formula will generate the necessary fragment of the Knowledge
Base. However, this is still not
very convenient.] \par
\leftskip=-3em
\item\textit{Installing the user interface component} \par
\leftskip=-1em
\item[$\Rightarrow$] \quad
\textit{sc-agent*:} \par
\leftskip=0pt
[ScComponentManagerInstallAgent]
\leftskip=-1em
\item[$\Rightarrow$] \quad
\textit{result*:} \par
\leftskip=0pt
[Interface component for logic output component installed] \par
\leftskip=-1em
\item[$\Rightarrow$] \quad
\textit{note*:} \par
\leftskip=0pt
[After performing this step in the
web interface after creating the
necessary fragment to send the
formula on geometry, you can easily call the logical output agent
through the interface component.] \par
\leftskip=-4em
\rangle \par
\leftskip=-2em
\item[$\Rightarrow$] \quad
\leftskip=0pt
\setlength { \columnsep }{ 0.5cm }
\textit{result*:}
[The functionality of the system is extended. A system capable of logical inference and finding intersection of sets is obtained. The system has interface components corresponding to these agents.
The Knowledge Base on geometrical figures (polygons and triangles) is also obtained.]
\begin{center}
    Conclusion
\end{center}
\leftskip=-4em
The component approach is key in the technology
of designing intelligent computer systems. At the same
time, the technology of component design is closely
related to the other components of the technology of
designing intelligent computer systems and ensures their
compatibility, producing a powerful synergetic effect
when using the entire complex of private technologies for
designing intelligent systems. The most important principle in the implementation of the component approach
is the semantic compatibility of reusable components,
which minimizes the participation of programmers in the
creation of new computer systems and the improvement
of existing ones.

\newpage
\\
To implement the component approach, in the article, a
library of reusable compatible components of intelligent
computer systems based on the OSTIS Technology is
proposed, classification and specification of reusable
ostis-systems components is introduced, a component
manager model is proposed that allows ostis-systems
to interact with libraries of reusable components and
manage components in the system, the architecture of the
ecosystem of intelligent computer systems is considered
from the point of view of using a library of reusable
components. \par
At the moment the manager of reusable components of
ostis-systems with console user interface and the library
of reusable components of ostis-systems with graphical
user interface have been implemented. The subject areas
necessary for the implementation of component design
have been implemented, and diagrams showing the details of the use and operation of the component manager
and the component library have been implemented. \par
The results obtained will improve the design efficiency of intelligent systems and automation tools for
the development of such systems, as well as provide an
opportunity not only for the developer but also for the
intelligent system to automatically supplement the system
with new knowledge and skills.
\begin{center}
    References
\end{center}
\begin{enumerate}[label=\textit{[\arabic*]}]
\scriptsize
\leftskip=-10em
 \item K. Yaghoobirafi and A. Farahani, “An approach for semantic interoperability in autonomic distributed intelligent systems,” Journal
of Software: Evolution and Process, vol. 34, 02 2022.
 \item Natalia N. Skeeter, Natalia V. Ketko, Aleksey B. Simonov,
Aleksey G. Gagarin, Irina Tislenkova, “Artificial intelligence:
Problems and prospects of development,” Artificial Intelligence:
Anthropogenic Nature vs. Social Origin, 2020.
 \item Olena Yara, Anatoliy Brazheyev, Liudmyla Golovko, Liudmyla
Golovko, Viktoriia Bashkatova, “Legal regulation of the use
of artificial intelligence: Problems and development prospects,”
European Journal of Sustainable Development, 2021.
 \item J. Waters, B. J. Powers, and M. G. Ceruti, “Global interoperability using semantics, standards, science and technology (gis3t),”
Computer Standards \& Interfaces, vol. 31, no. 6, pp. 1158–1166,
2009.
 \item M. Wooldridge, An introduction to multiagent systems, 2nd ed.
Chichester : J. Wiley, 2009.
 \item X. Shi, Z. Zheng, Y. Zhou, H. Jin, L. He, B. Liu, and Q.-S.
Hua, “Graph processing on GPUs,” ACM Computing Surveys,
vol. 50, no. 6, pp. 1–35, Nov. 2018. [Online]. Available:
https://doi.org/10.1145/3128571
 \item P. Lopes de Lopes de Souza, W. Lopes de Lopes de Souza, and
R. R. Ciferri, “Semantic interoperability in the internet of things:
A systematic literature review,” in ITNG 2022 19th International
Conference on Information Technology-New Generations, S. Latifi, Ed. Cham: Springer International Publishing, 2022, pp. 333–
340.
 \item Blähser, Jannik and Göller, Tim and Böhmer, Matthias, “Thine
— approach for a fault tolerant distributed packet manager based
on hypercore protocol,” in 2021 IEEE 45th Annual Computers,
Software, and Applications Conference (COMPSAC), 2021, pp.
1778–1782.
 \item V. Torres da Silva, J. S. dos Santos, R. Thiago, E. Soares, and
L. Guerreiro Azevedo, “OWL ontology evolution: understanding
and unifying the complex changes,” The Knowledge Engineering
Review, vol. 37, p. e10, 2022.

\columnbreak

 \item V. Gribova,L. Fedorischev, P. Moskalenko, V. Timchenko, “Interaction of cloud services with external software and its implementation on the IACPaaS platform,” pp. 1–11, 2021.
\item Vladimir Golenkov and Natalia Guliakina and Daniil Shunkevich,
Open technology of ontological design, production and operation
of semantically compatible hybrid intelligent computer systems,
V. Golenkov, Ed. Minsk: Bestprint [Bestprint], 2021.
\item (2022, September) IMS.ostis Metasystem. [Online]. Available:
https://ims.ostis.net
\item Shunkevich D.V., Davydenko I.T., Koronchik D.N., Zukov
I.I., Parkalov A.V., “Support tools knowledge-based systems
component design,” Open semantic technologies for intelligent
systems, pp. 79–88, 2015. [Online]. Available: http://proc.ostis.
net/proc/Proceedings%20OSTIS-2015.pdf
\item G. Sellitto, E. Iannone, Z. Codabux, V. Lenarduzzi, A. De Lucia,
F. Palomba, and F. Ferrucci, “Toward understanding the impact
of refactoring on program comprehension,” in 29th International
Conference on Software Analysis, Evolution, and Reengineering
(SANER), 2022, pp. 1–12.
\item M. K. Orlov, “Comprehensive library of reusable semantically
compatible components of next-generation intelligent computer
systems,” in Open semantic technologies for intelligent systems,
ser. Iss. 6. Minsk : BSUIR, 2022, pp. 261–272.
\item ——, “Control tools for reusable components of intelligent computer systems of a new generation,” in Open semantic technologies for intelligent systems, ser. Iss. 7. Minsk : BSUIR, 2023,
pp. 261–272.
\end{enumerate}
\normalsize
\begin{center}
\textbf{ТЕКУЩЕЕ СОСТОЯНИЕ СРЕДСТВ
АВТОМАТИЗАЦИИ КОМПОНЕНТНОГО
ПРОЕКТИРОВАНИЯ OSTIS-СИСТЕМ} \par
Орлов М. К., Макаренко А. И.,
Петрочук К. Д. 

\end{center}

В работе рассматривается подход к проектированию
интеллектуальных систем, ориентированный на использование совместимых многократно используемых
компонентов, что существенно сокращает трудоемкость разработки таких систем. Ключевым средством
поддержки компонентного проектирования интеллектуальных компьютерных систем является предложенный в работе менеджер многократно используемых
компонентов.
\begin{flushright}
    Received 03.03.2024
\end{flushright}




 \end{itemize}
\end{multicols}
\end{document}

\documentclass[a4paper]{article}
\usepackage[T2A]{fontenc}
\usepackage[utf8]{inputenc}
\usepackage[english, russian]{babel}
\usepackage{multicol}
\usepackage[left=2cm,right=2cm, top=2cm,bottom=2cm]{geometry}
\usepackage{enumitem}
\usepackage{graphicx}
\usepackage{caption}
\usepackage[unicode, pdftex]{hyperref} % Пакет для подключения ссылок


\begin{document}
\setcounter{page}{186}
\begin{multicols}{2}

\begin{enumerate}[label=\arabic*)]
\itemsep=-1.5mm
\item Railway lines and stations: This subject area describes information about various railway lines,
including their geographical location, length, speed
limits, track types, and accessibility. It also includes information about stations, including their location, capacity, and cargo and passenger handling capabilities.
\item Cargoes and containers: This subject area describes various types of cargoes and containers, their characteristics (weight, volume, type of packaging), processing and storage requirements. It also includes information about special requirements for
certain types of goods, for example, goods requiring special transportation conditions (temperature conditions, vibration protection, etc.).
 \item Schedule and Train Timetable: This subject area
describes information about the train timetable,
including departure and arrival times, intermediate stops, travel speeds and possible overlaps on
the track. It also includes information about the
regularity and repeatability of the train schedule.
\item Resources: This subject area describes information about available resources such as locomotives, wagons, personnel, and technical equipment.
It includes the characteristics of resources (load capacity, technical parameters), their availability and maintenance schedule.
\item Laws and Regulations: This subject area describes
the laws, rules and regulations governing railway
logistics. It includes safety rules, standards for the
transportation of certain goods, requirements for
staff working hours, rules for the priority of train
traffic and other regulatory information.
\item Monitoring and Data collection: This subject area
describes the monitoring and data collection systems used to obtain information about the current
state of railway logistics. It includes data on train
movement, cargo status, information about delays
and other factors affecting planning and management.
\end{enumerate}

\par Various problem solvers can be used to automate
transportation planning and management. Here are some
typical problem solvers that can be applied in this field:

\begin{enumerate}[label=\arabic*)]
\itemsep=-1.5mm
\item The routing problem solver: This solver allows
you to optimize train routes, taking into account
various factors such as schedules, availability of
railway lines, cargo requirements and restrictions
on infrastructure and transportation conditions. It
helps to choose the best routes, minimizing the
time and cost of transportation.
\item Train timetable development solver: This solver
allows to optimize the train timetable, taking into
account the timetable, passenger and cargo requirements, infrastructure constraints and other factors. It helps to allocate trains by time and resources,ensuring efficient use of the railway network.
\item Loading planning problem solver: This solver helps
to optimize the loading of wagons, taking into
account the characteristics of goods and limitations
of wagons, such as load capacity, dimensions and
special requirements. It helps to maximize the
loading of wagons, minimizing empty runs and
improving the use of the wagon fleet.
\item The solver of the optimal resource planning problem: This solver allows you to optimize the allocation of resources such as locomotives, wagons,personnel and technical equipment to ensure the efficient operation of the railway system. It takes into
account transportation requirements, traffic schedules and resource constraints, helping to achieve
optimal use of resources and reduce costs.
 \item Demand Forecasting and Planning Solver: This
solver is used to predict the demand for rail
transportation, taking into account various external
factors such as economic conditions, seasonality,
trends and others. It helps to plan capacity and
resources to meet demand and prevent congestion
or lack of transportation capacity.
\item Monitoring and Management Problem Solver: This
solver is used for continuous monitoring and management of railway transportation. It analyzes data
on train movements, traffic status, delays and other
events, allowing you to make operational decisions
and respond to changes in real time.
\end{enumerate} 
\par These are just some examples of problem solvers that
can be used to automate the planning and management of
railway logistics. The specific set of solvers will depend
on the specific requirements and characteristics of the
railway logistics system, as well as on the goals and
constraints set.
\par Using ontology for these problem solvers in the field
of railway logistics can bring several advantages:

\begin{enumerate}[label=\arabic*)]
\itemsep=-1.5mm
\item Knowledge structuring: Ontology allows you to
structure knowledge about the main subject areas of railway logistics, such as infrastructure,
resources, cargo and schedules. This facilitates
the understanding and organization of information,
simplifies its accessibility and exchange between
different systems and participants.
\item Data unification and standardization: An ontology
defines common semantics and standards for data
representation in railway logistics. This allows different systems and applications to use the same terms and data formats, which ensures consistency
and compatibility of information. This is especially
important when integrating different systems and
exchanging data between them.
\item Improvement of planning and optimization: Ontology provides a formalized domain model on the basis of which planning and optimization algorithm can be developed. The use of ontology makes it possible to take into account various limitations,
requirements and dependencies between different
aspects of railway logistics. This contributes to
more efficient use of resources, route optimization
and improved service quality.
\item Decision Making Improvement: Ontology can be
used in decision support systems, providing a semantic model and context for data analysis and
recommendation generation. It allows systems to
identify dependencies and relationships between
various factors, to carry out forecasting and scenario analysis, which helps to make reasonable and
informed decisions.
\item  Ease of expansion and modification: Ontology
provides a flexible and extensible domain model
that can be easily modified and supplemented as
needed. This allows you to adapt the ontology to
new requirements, changing conditions and expand
its functionality. This flexibility makes it easier to
support different types of tasks and to develop the
system in the future
\end{enumerate}

\par In general, the use of ontology for data solvers in the
field of railway logistics contributes to improving data
organization, information compatibility and consistency,
optimizing planning and management processes, as well
as making informed decisions based on data analysis.
\par The ontology for these problem solvers in the field
of transportation process management should be flexible
and modular in order to take into account various factors
and adapt to changing conditions.

\begin{figure}
\includegraphics{фотка1.eps}
{Figure 2. Semantic neighborhood of knowlegebase RW }
\label{fig:фотка1}
\end{figure}

\par\quad Within the framework of this work, a top-level ontology is implemented describing the main objects of the
transportation process, examples of objects of which are
presented in Figures \ref{fig:фотка1} 2-4 in the form of an SCg code [10].

\begin{center}
\MakeUppercase {\romannumeral 5}. Conclusion
\end{center}

\par Thus, within the framework of this work, the relevance
of developing an ontology for an intelligent transportation process management system has been determined.

\includegraphics{фотка2.png}
{Figure 3. Semantic neighborhood of static objects RW}


\includegraphics{фотка3.png}
{Figure 4. Semantic neighborhood of dynamic objects RW}

\par The structure of the theory of ontology construction
is given. The existing ontologies of railway systems
are described. The problems of existing ontologies are
established. It is proposed to use a process-object approach in the formation of ontology. Examples of the
use of ontology are given. It is indicated that the OSTIS
technology is an effective tool for describing the processobject ontology of the transportation process. The toplevel ontology for the Belarusian Railway has been implemented.
 
\centering\renewcommand{\refname}{References}
\begin{thebibliography}{10}
\itemsep=-1.5mm
\bibitem A. A. Erofeev Intelectualnaya sistema upravlenia perevozochnim
processom [Intelligent control system for the transportation process in railway transport]. The Ministry of Education of the Republic. Belarus, Belarusian state University of Transport. Gomel,
BSUT, 2022, P. 407
\bibitem A. P. Badeckii, O. A. Medved Ontologicheskii podhod k
razrabotke edinoi basi znanii multimodalnih perevozok [An ontological approach to the development of a unified knowledge base for multimodal transportation]. Proceedings of the St. Petersburg University of Railway Engineering, 2023, № 1, pp. 182–193
\bibitem 
Ontology Description Capture IDEF5. https://www.idef.
com/idef5-ontology-description-capture-method/ (Accessed
10.03.2024)
\bibitem T. Gruber Toward Principles for the Design of Ontologies Used
for Knowledge Sharing. Int. Journal Human-Computer Studies,
Vol. 43, pp. 907–928.
\bibitem A. D. Obuhov Ontology neshtatnih situacii na sortirovochnoi
stancii po nepriemu poezda [Ontology of emergency situations
at the marshalling yard because of unreception of trains]. Materials of the IV International Scientific and Practical Conference
"Modern concepts of scientific researchers", N. Novgorod, 2015,
pp. 281-284.
\bibitem A. V. Borisov, A. V.Bosov, D. V.Zukov, A. V. Ivanov, D. V.
Sushko, Informationnie aspekti obespechenia bezopasnosti na
transporte: ontologiya predmetnoy oblasti, modeli i varianti ispolzovania [Information aspects of transport safety: domain ontology, models and use cases], Systems and Means of Informatics,
Federal Research Center "Computer Science and Control" of the
Russian Academy of Sciences: 30:1 (2020), pp. 126–134
\bibitem V. Golenkov and N. Guliakina, “Principles of building mass
semantic technology component design of intelligent systems,”
in Open semantic technologiesfor intelligent systems. Minsk,
BSUIR, 2011, pp. 21–58 (in Russian)
\bibitem V. Golenkov and N. Gulyakina, “Semanticheskaya tekhnologiya
komponentnogo proektirovaniya sistem, upravlyaemyh znaniyami
[Semantic technology of component design of knowledge-driven
systems],” in Open semantic technologiesfor intelligent systems,
Minsk, BSUIR, 2015, pp. 57–78, (in Russian)
\bibitem  V. V. Golenkov, “Ontology-based design of intelligent systems,”
in Open semantic technologies for intelligent systems, ser. Iss. 1.
Minsk : BSUIR, 2017, pp. 37–56.
\bibitem V. V.Golenkov, Tehnologija kompleksnoj podderzhki zhiznennogo
cikla semanticheski sovmestimyh intellektual’nyh komp’juternyh
sistem novogo pokolenija [Technology of complex life cycle
support of semantically compatible intelligent computer systems
of new generation ] – Minsk:Bestprint, 2023. P. 1063 (in Russian)
\end{thebibliography}

\begin{center}
\par\textbf{ОСНОВЫ ОНТОЛОГИИ ПЕРЕВОЗОЧНОГО ПРОЦЕССА НА ЖЕЛЕЗНОДОРОЖНОМ ТРАНСПОРТЕ}
\end{center}

\begin{center}
{Ерофеев А. А., Ерофеев И. А.}
\end{center}

\par Определена актуальность разработки онтологии для интеллектуальной системы управления перевозочным
процессом. Приведена структура теории построения онтологии. Описаны существующие онтологии железнодорожных систем. Установлены проблемы существующих онтологий. Предложено при формировании онтологии использовать процессно-объектный подход. Приведены примеры использования онтологии. Указано, что эффективным инструментом описания процессно-объектной онтологии перевозочного процесса является технология OSTIS.

\begin{flushright}
\par Received 13.03.2024
\end{flushright}

\end{multicols}

\end{document}


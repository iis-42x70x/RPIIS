\documentclass[a4paper,12pt]{article}
\usepackage[utf8]{inputenc}
\usepackage[russian]{babel}
\usepackage{amsmath}
\usepackage{amssymb}
\usepackage{geometry}
\geometry{left=2.5cm,right=2.5cm,top=2cm,bottom=2cm}

\title{Комплексные числа}
\author{ }
\date{ }

\begin{document}

\maketitle

\section{Введение}
Комплексные числа — это числа, которые можно представить в виде \( z = a + bi \), где \( a \) и \( b \) — действительные числа, а \( i \) — мнимая единица, такая, что \( i^2 = -1 \). Рассмотрим основные представления и операции с комплексными числами.

\section{Алгебраическое представление}
Алгебраическая форма комплексного числа — это его запись в виде:
\[
z = a + bi
\]
Здесь \( a \) — действительная часть, а \( b \) — мнимая часть. Например, число \( 3 + 4i \) имеет действительную часть \( a = 3 \) и мнимую часть \( b = 4 \).

Операции в алгебраической форме:
\begin{itemize}
    \item Сложение: \( (a + bi) + (c + di) = (a + c) + (b + d)i \)
    \item Умножение: \( (a + bi)(c + di) = (ac - bd) + (ad + bc)i \)
    \item Модуль: модуль комплексного числа \( z = a + bi \) определяется как:
    \[
    |z| = \sqrt{a^2 + b^2}
    \]
    \item Комплексно-сопряжённое: Комплексное число, сопряжённое с \( z = a + bi \), обозначается как \( \overline{z} = a - bi \).
\end{itemize}

\section{Геометрическое представление}
Комплексные числа можно интерпретировать как точки на комплексной плоскости. Ось \( x \) (горизонтальная) соответствует действительной части числа, а ось \( y \) (вертикальная) — мнимой части. Таким образом, комплексное число \( z = a + bi \) можно изобразить как точку \( (a, b) \).

\subsection{Полярные координаты}
Для работы с комплексными числами часто удобно использовать полярные координаты. В полярной форме число \( z \) задаётся модулем \( r = |z| \) и углом \( \theta \), который число образует с положительным направлением оси \( x \). Тогда \( z \) можно выразить как:
\[
z = r (\cos \theta + i \sin \theta)
\]
Здесь:
\[
r = \sqrt{a^2 + b^2}, \quad \theta = \arg(z) = \tan^{-1} \left( \frac{b}{a} \right)
\]

\section{Тригонометрическое представление}
Комплексное число также можно записать в тригонометрической форме:
\[
z = r \left( \cos \theta + i \sin \theta \right)
\]
Иногда это представление записывают с использованием формулы Эйлера:
\[
z = r e^{i \theta}
\]
где \( e^{i \theta} = \cos \theta + i \sin \theta \).

\section{Возведение в степень}
Комплексные числа легко возводить в степень, если они представлены в тригонометрической форме. Пусть \( z = r (\cos \theta + i \sin \theta) \), тогда для любого целого числа \( n \):
\[
z^n = r^n \left( \cos (n \theta) + i \sin (n \theta) \right)
\]
Это известная формула Муавра.

\section{Извлечение корня}
Для извлечения корня из комплексного числа используется тригонометрическая форма. Пусть \( z = r (\cos \theta + i \sin \theta) \), тогда \( n \)-ый корень из числа \( z \) имеет вид:
\[
z^{1/n} = \sqrt[n]{r} \left( \cos \frac{\theta + 2k\pi}{n} + i \sin \frac{\theta + 2k\pi}{n} \right), \quad k = 0, 1, 2, \dots, n-1
\]
Таким образом, у каждого комплексного числа существует \( n \) различных корней.

\section{Пример}
Возьмём число \( z = 1 + i \). Найдём его в разных представлениях и вычислим корни.
\begin{itemize}
    \item Алгебраическая форма: \( z = 1 + i \).
    \item Модуль: \( r = \sqrt{1^2 + 1^2} = \sqrt{2} \).
    \item Аргумент: \( \theta = \tan^{-1}(1/1) = \frac{\pi}{4} \).
    \item Тригонометрическая форма: \( z = \sqrt{2} \left( \cos \frac{\pi}{4} + i \sin \frac{\pi}{4} \right) \).
    \item Возведение в квадрат: \( z^2 = 2 \left( \cos \frac{\pi}{2} + i \sin \frac{\pi}{2} \right) = 2i \).
    \item Извлечение квадратного корня: \( z^{1/2} = \sqrt{\sqrt{2}} \left( \cos \frac{\pi}{8} + i \sin \frac{\pi}{8} \right) \).
\end{itemize}

Эти представления и операции с комплексными числами полезны в решении различных задач, таких как работа с электрическими цепями, анализ сигналов и квантовая механика.

\end{document}
